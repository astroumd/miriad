\documentclass{article}
\usepackage{html}
\usepackage{graphicx}

% Softwares
\newcommand{\MIRIAD}{\htmladdnormallink{\texttt{MIRIAD}}{http://bima.astro.umd.edu/miriad/}}
\newcommand{\GILDAS}{\htmladdnormallink{\texttt{GILDAS}}{http://www.iram.fr/IRAMFR/GS/gildas.htm}}
\newcommand{\python}{\htmladdnormallink{\texttt{python}}{http://www.python.org/}}
\newcommand{\fpig}{\htmladdnormallink{\texttt{f2py}}{http://cens.ioc.ee/projects/f2py2e/}}
\newcommand{\numpy}{\htmladdnormallink{\texttt{numpy}}{http://www.pfdubois.com/numpy/}}

% Instruments
\newcommand{\PdBI}{\htmladdnormallink{PdBI}{http://www.iram.fr/IRAMFR/index.htm}}
\newcommand{\IRAMthm}{\htmladdnormallink{IRAM-30m}{http://www.iram.fr/IRAMES/index.htm}}
\newcommand{\CSO}{\htmladdnormallink{CSO}{http://www.submm.caltech.edu/cso/}}

\title{First steps to interoperate \GILDAS{} and \MIRIAD{}\\[3\bigskipamount]}

\author{%
  J.~Pety~(1,2), F.~Gueth~(2), S.~Guilloteau~(2,3), R.~Lucas~(2), \\
  P.~Teuben~(4) and M.~Wright~(5) \\[0.5cm]
  1. \htmladdnormallink{LERMA}{http://aramis.obspm.fr/lerma.shtml}, \htmladdnormallink{Observatoire de Paris}{http://www.obspm.fr/obsparis.fr.shtml}\\
  2. \htmladdnormallink{IRAM}{http://www.iram.fr/index.htm} (Grenoble)\\
  3. Observatoire de Bordeaux\\
  4. Astronomy Department, U. Maryland\\
  5. Radio Astronomy Laboratory, U.C. Berkeley\\[3\bigskipamount]}

%%%%%%%%%%%%%%%%%%%%%%%%%%%%%%%%%%%%%%%%%%%%%%%%%%%%%%%%%%%%%%%%%%%%%%%%%%%

\begin{document}

\maketitle{} %

\GILDAS{} and \MIRIAD{} are two state-of-the-art data reduction packages
for the current generation of millimeter instruments. \GILDAS{} is used
daily at the Plateau de Bure Interferometer (\PdBI{}) as well as several
single dish telescopes (e.g. \IRAMthm{}, \CSO{}) while \MIRIAD{} is used at
BIMA, ATCA, OVRO and WSRT.  Although the core functionalities needed to
reduce millimeter interferometry data are well covered in both packages, it
has recently been recognized that the strength of both packages are
complementary.  \GILDAS{} has good data analysis programs while \MIRIAD{}
has a complete set of calibration and imaging algorithms in particular to
handle polarization. It is thus interesting to study the possibility of
interoperating both packages in a user-friendly way. Both packages have
pretty different implementation philosophy: \GILDAS{} is principally made
of large stand-alone programs while \MIRIAD{} is built around a large
collection of individual tasks. Moreover, they use different command line
interfaces and, in detail, data models. In a first step toward
interoperability, we decided to run both packages under python and to
exchange data using FITS.

\section{Installation}
\label{sec:installation}

Total duration: about ??~h. As all the other time given, this is just an
indication. It just takes into account the compilation on a Athlon PC with
756~MB of RAM.

\subsection{Needed parts}

\begin{itemize}
\item \python{} + \fpig{} + \numpy{} (Numeric Python is needed by \fpig{})
\item \GILDAS{} (GNU make, Intel f90, LessTif)
\item \MIRIAD{}  
\end{itemize}

For your convenience, the version of the softs used for this test may be
download here: %
\htmladdnormallink{\texttt{python}}{src/Python-2.2.2.tar.gz}, %
\htmladdnormallink{\texttt{f2py}}{src/F2PY-2.32.225-1419.tar.gz}, %
\htmladdnormallink{\texttt{numpy}}{src/Numeric-22.0.tar.gz}, %
\htmladdnormallink{\texttt{GILDAS}}{src/gildas-oct03a.tar.gz}, %
\htmladdnormallink{\texttt{MIRIAD}}{src/miriad-4.0.3.tar.gz}.
The cumulative size of those software is 28~MB.

\subsection{Preparation}

\begin{verbatim}
  pctcp104> mkdir $HOME/gildas-miriad
  pctcp104> cd $HOME/gildas-miriad
\end{verbatim}

\subsection{Installation of \python{} (about ?? min.)}

For your convenience, the version of \python{} used for this test may be
download \htmladdnormallink{here}{src/Python-2.2.2.tar.gz} (6.4~MB).

\begin{verbatim}
  pctcp104> cd $HOME/gildas-miriad
  pctcp104> tar -xzvf Python-2.2.2.tar.gz
  pctcp104> cd Python-2.2.2
  ~(pctcp104> more README)
  pctcp104> mkdir local  
  pctcp104> ./configure prefix=$HOME/gildas-miriad/local
  pctcp104> make
  pctcp104> make altinstall
  pctcp104> export PATH=$PATH:$HOME/gildas-miriad/local/bin
  pctcp104> python2.2
  >>> help~()
  >>> CTRL-D 
  pctcp104> cd ..
\end{verbatim}

\subsection{Installation of \numpy{} (about ?? min.)}

For your convenience, the version of \numpy{} used for this test may be
download \htmladdnormallink{here}{src/Numeric-22.0.tar.gz} (712~kB).

\begin{verbatim}
  pctcp104> cd $HOME/gildas-miriad
  pctcp104> tar -xzvf Numeric-22.0.tar.gz
  pctcp104> cd Numeric-22.0
  pctcp104> more README
  pctcp104> python2.2 setup.py install
  pctcp104> python2.2
  >>> import Numeric
  >>> help~(Numeric)
  >>> CTRL-D
\end{verbatim} %$

\subsection{Installation of \fpig{} (about ?? min.)}

For your convenience, the version of \fpig{} used for this test may be
download \htmladdnormallink{here}{src/F2PY-2.32.225-1419.tar.gz} (152~kB).

\begin{verbatim}
  pctcp104> cd $HOME/gildas-miriad
  pctcp104> tar -xzvf F2PY-2-latest.tar.gz
  pctcp104> cd F2PY-2.32.225-1419
  pctcp104> more docs/README
  pctcp104> python2.2 setup.py install
  pctcp104> f2py2.2  -m hello -c hello.f
  pctcp104> python2.2
  >>> import hello
  >>> help~(hello)
  >>> print hello.__doc__
  This module 'hello' is auto-generated with f2py ~(version:2.32.225-1419).
  Functions:
    foo~(a).
  >>> hello.foo~(44)
   Hello from Fortran!
   a=          44
  >>> CTRL-D
\end{verbatim} %$

\subsection{Installation of \GILDAS{} (about ?? min.)}

For your convenience, the version of \GILDAS{} used for this test may be
download \htmladdnormallink{here}{src/gildas-oct03a.tar.gz} (14~MB).

\begin{verbatim}
  pctcp104> cd $HOME/gildas-miriad
  pctcp104> bash
  pctcp104> tar -xzvf gildas-oct03a.tar.gz
  pctcp104> cd gildas-oct03a
  pctcp104> source admin/gildas-env.sh
  pctcp104> make python
  pctcp104> make install-python
  pctcp104> source $HOME/gildas-miriad/gildas-exe/pc-redhat7.2-ifc/bash_profile
\end{verbatim}

Alternatively you may want to directly access the CVS repository:
\begin{verbatim}
  pctcp104> cd $HOME/gildas-miriad
  pctcp104> bash
  pctcp104> export CVSROOT=:pserver:anonymous@netsrv1.iram.fr:/CVS/GILDAS
  pctcp104> cvs co -r oct03a -d gildas-oct03a gildas
  pctcp104> cd gildas-oct03a
  pctcp104> source admin/gildas-env.sh
  pctcp104> make python
  pctcp104> make install-python
  pctcp104> source $HOME/gildas-miriad/gildas-exe/pc-redhat7.2-ifc/bash_profile
\end{verbatim}

\subsection{Installation of \MIRIAD{} (about ?? min.)}

For your convenience, the version of \MIRIAD{} used for this test may be
download \htmladdnormallink{here}{src/miriad-4.0.3.tar.gz} (7.1~MB).

\begin{verbatim}
  pctcp104> cd $HOME/gildas-miriad
  pctcp104> tar -xzvf miriad-4.0.3.tar.gz
  pctcp104> cd miriad/install
  pctcp104> ./install.linux
  pctcp104> source ../MIRSRC.linux.good
  pctcp104> export PYTHONPATH=$PYTHONPATH:$MIR/src/scripts/python
\end{verbatim} %$

Alternatively you may want to directly access the CVS repository:
\begin{verbatim}
  pctcp104> cd $HOME/gildas-miriad
  pctcp104> export CVSROOT=:pserver:anonymous@cvs.astro.umd.edu:/home/cvsroot
  pctcp104> cvs co miriad
  pctcp104> cd miriad/install
  pctcp104> ./install.linux
  pctcp104> source ../MIRSRC.linux
  pctcp104> export PYTHONPATH=$PYTHONPATH:$MIR/src/scripts/python
\end{verbatim} %$

\section{Demonstration}
\label{sec:demo}

Raw data from \PdBI{} easily amount to hundred of MB. It has thus been
decided to leave the choice between two demonstration:
\begin{itemize}
\item In the first one, we start from the calibrated visibilities in uvfits
  format. The imaging and deconvolution steps are made inside \MIRIAD{} and
  the visualization inside \GILDAS{}.  
\item In the second one, we start from the raw visibilities in native
  \PdBI{} format. Visibilities are calibrated inside \GILDAS{}, the imaging
  and deconvolution steps are made inside \MIRIAD{} and the visualization
  inside \GILDAS{}. All communication between \GILDAS{} and \MIRIAD{} is
  done through FITS.
\end{itemize}
In both cases, both packages are called from the same python process.

\subsection{Shell environment}

To be ready to run the demonstration, we installed (\ref{sec:installation})
no less than five different softwares. If you leaved the xterm you were
working in, you have lost all the needed shell environment. So we gathered
here what is mandatory to allow the demonstrations to run smoothly. 
\begin{verbatim}
  pctcp104> export PATH=$PATH:$HOME/gildas-miriad/local/bin                    
  pctcp104> source $HOME/gildas-miriad/gildas-exe/pc-redhat7.2-ifc/bash_profile
  pctcp104> source $HOME/gildas-miriad/miriad/MIRSRC.linux                     
  pctcp104> export PYTHONPATH=$PYTHONPATH:$MIR/src/scripts/python                   
\end{verbatim}
Those probably need a bit of adaptation to your own installation. For
instance, \texttt{pc-redhat-7.2-ifc} may not be the system you are working
on. To know what is correct for you, just look in the
\texttt{\$HOME/gildas-miriad/gildas-exe} directory.

\subsection{Partial demonstration}

The data and script needed to run the partial demonstration are available
as a gzip tarball \htmladdnormallink{here}{demo/partdemo.tar.gz} (7.4~MB).

\begin{verbatim}
  pctcp104> cd $HOME/gildas-miriad
  pctcp104> tar -xzvf partdemo.tar.gz
  pctcp104> cd partdemo
  pctcp104> ./partdemo.py
\end{verbatim} %$
If you want to come back to the orginal state, just type \texttt{make
  clean} at the prompt.

\subsection{Full demonstration}

Right now only the calibration inside \GILDAS{} is implemented.

The data and script needed to run the partial demonstration are available
as a gzip tarball \htmladdnormallink{here}{demo/fulldemo.tar.gz}
(790~MB).

\begin{verbatim}
  pctcp104> cd $HOME/gildas-miriad
  pctcp104> tar -xzvf fulldemo.tar.gz
  pctcp104> cd fulldemo
  pctcp104> ./fulldemo.py
\end{verbatim} %$
If you want to come back to the orginal state, just type \texttt{make
  clean} at the prompt.

\end{document}

%%%%%%%%%%%%%%%%%%%%%%%%%%%%%%%%%%%%%%%%%%%%%%%%%%%%%%%%%%%%%%%%%%%%%%%%%%%

%\begin{htmlonly}
%\begin{rawhtml}
%<BODY BACKGROUND="bure3.jpeg">
%\end{rawhtml}
%\end{htmlonly}

\begin{htmlonly}
\htmladdftplink{The FITS file.}{Images/turbulence.fits}
\end{htmlonly}

\begin{figure}
\includegraphics[width=10.0cm,angle=270.0]{Images/turbulence.eps}
\end{figure}

%%%%%%%%%%%%%%%%%%%%%%%%%%%%%%%%%%%%%%%%%%%%%%%%%%%%%%%%%%%%%%%%%%%%%%%%%%%
