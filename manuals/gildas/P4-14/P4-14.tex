%APN3_PROCEEDINGS_FORM%%%%%%%%%%%%%%%%%%%%%%%%%%%%%%%%%%%%%%%%%%%%%%%
%
% TEMPLATE.TEX -- APN3 (2003) ASP Conference Proceedings template.
%
% Derived from ADASS VIII (98) ASP Conference Proceedings template
% Updated by N. Manset for ADASS IX (99), F. Primini for ADASS 2000,
% D.Bohlender for ADASS 2001, and H. Payne for ADASS XII and LaTeX2e.
%
% Use this template to create your proceedings paper in LaTeX format
% by following the instructions given below.  Much of the input will
% be enclosed by braces (i.e., { }).  The percent sign, "%", denotes
% the start of a comment; text after it will be ignored by LaTeX.  
% You might also notice in some of the examples below the use of "\ "
% after a period; this prevents LaTeX from interpreting the period as
% the end of a sentence and putting extra space after it.  
% 
% You should check your paper by processing it with LaTeX.  For
% details about how to run LaTeX as well as how to print out the User
% Guide, consult the README file.  You should also consult the sample
% LaTeX papers, sample1.tex and sample2.tex, for examples of including
% figures, html links, special symbols, and other advanced features.
%
% If you do not have access to the LaTeX software or a laser printer
% at your site, you can still prepare your paper following the
% instructions in the User Guide.  In such cases, the editors will
% process the file and make any necessary editorial adjustments.
% 
%%%%%%%%%%%%%%%%%%%%%%%%%%%%%%%%%%%%%%%%%%%%%%%%%%%%%%%%%%%%%%%%%%%%%%%%
% 
\documentclass[11pt,twoside]{article}  % Leave intact
\usepackage{adassconf}

% If you have the old LaTeX 2.09, and not the current LaTeX2e, comment
% out the \documentclass and \usepackage lines above and uncomment
% the following:

%\documentstyle[11pt,twoside,adassconf]{article}

% Software
\newcommand{\FORTRAN}{\texttt{FORTRAN}}
\newcommand{\AIPSpp}{\texttt{AIPS$++$}}
\newcommand{\GILDAS}{\texttt{GILDAS}}
\newcommand{\MIRIAD}{\texttt{MIRIAD}}
\newcommand{\python}{\texttt{python}}
\newcommand{\fpig}{\texttt{f2py}}
\newcommand{\numpy}{\texttt{numpy}}

%
\newcommand{\SIC}{\texttt{SIC}}
\newcommand{\GREG}{\texttt{GREG}}
\newcommand{\CLASS}{\texttt{CLASS}}
\newcommand{\NIC}{\texttt{NIC}}
\newcommand{\CLIC}{\texttt{CLIC}}
\newcommand{\MAPPING}{\texttt{MAPPING}}

% Instruments
\newcommand{\ALMA}{\textrm{ALMA}}
\newcommand{\IRAM}{\textrm{IRAM}}
\newcommand{\IRAMthm}{\textrm{IRAM-30m}}
\newcommand{\PdBI}{\textrm{PdBI}}
\newcommand{\CSO}{\textrm{CSO}}

%
\newcommand{\ie} {{\em i.e.}}
\newcommand{\eg} {{\em e.g.}}
\newcommand{\emm}[1]{\ensuremath{#1}} % Ensures math mode.
\newcommand{\implication}{\emm{\Rightarrow}}

\newcommand{\thCO} {\mbox{$^{13}$CO}} % 13CO
\newcommand{\HCOp} {\mbox{HCO$^{+}$}} % HCO+
\newcommand{\Jone}{\mbox{(J=1--0)}}
\newcommand{\Jtwo}{\mbox{(J=2--1)}}

\begin{document}   % Leave intact

%-----------------------------------------------------------------------
%                           Paper ID Code
%-----------------------------------------------------------------------
% Enter the proper paper identification code.  The ID code for your
% paper is the session number associated with your presentation as
% published in the official conference proceedings.  You can           
% find this number locating your abstract in the printed proceedings
% that you received at the meeting or on-line at the conference web
% site; the ID code is the letter/number sequence proceeding the title 
% of your presentation. 
%
% This will not appear in your paper; however, it allows different
% papers in the proceedings to cross-reference each other.  Note that
% you should only have one \paperID, and it should not include a
% trailing period.
%
% EXAMPLE: \paperID{O4-1}
% EXAMPLE: \paperID{P7-7}
%

\paperID{P4-14}

%-----------------------------------------------------------------------
%                           Paper Title 
%-----------------------------------------------------------------------
% Enter the title of the paper.
%
% EXAMPLE: \title{A Breakthrough in Astronomical Software Development}
% 
% If your title is so long as to fill the page header when you print it,
% then please supply a short form as a \titlemark.
%
% EXAMPLE: 
%  \title{Rapid Development for Distributed Computing, with Implications
%         for the Virtual Observatory}
%  \titlemark{Rapid Development for Distributed Computing}
%

\title{Interoperating \GILDAS{} and \MIRIAD{}}

%\titlemark{}

%-----------------------------------------------------------------------
%                         Authors of Paper
%-----------------------------------------------------------------------
% Enter the authors followed by their affiliations.  The \author and
% \affil commands may appear multiple times as necessary (see example
% below).  List each author by giving the first name or initials first
% followed by the last name.  Authors with the same affiliations
% should grouped together. 
%
% EXAMPLE: \author{Raymond Plante, Doug Roberts, 
%                  R.\ M.\ Crutcher\altaffilmark{1}}
%          \affil{National Center for Supercomputing Applications, 
%                 University of Illinois Urbana-Champaign, Urbana, IL
%                 61801}
%          \author{Tom Troland}
%          \affil{University of Kentucky}
%
%          \altaffiltext{1}{Astronomy Department, UIUC}
%
% In this example, the first three authors, "Plante", "Roberts", and
% "Crutcher" are affiliated with "NCSA".  "Crutcher" has an alternate 
% affiliation with the "Astronomy Department".  The fourth author,
% "Troland", is affiliated with "University of Kentucky"

\author{J.~Pety\altaffilmark{1}}
\affil{Observatoire de Paris, France}
\author{F.~Gueth, S.~Guilloteau\altaffilmark{2}, R.~Lucas}
\affil{IRAM Grenoble, France}
\author{P.~J.~Teuben}
\affil{Astronomy Department, U. Maryland}
\author{M.~C.~H.~Wright}
\affil{Radio Astronomy Laboratory, U.C. Berkeley}

\altaffiltext{1}{IRAM Grenoble, France}
\altaffiltext{2}{Observatoire de Bordeaux, France}

%-----------------------------------------------------------------------
%                        Contact Information
%-----------------------------------------------------------------------
% This information will not appear in the paper but will be used by
% the editors in case you need to be contacted concerning your
% submission.  Enter your name as the contact along with your email
% address.
% 
% EXAMPLE:  \contact{Dennis Crabtree}
%           \email{crabtree@cfht.hawaii.edu}
%

\contact{J\'er\^ome Pety}
\email{pety@iram.fr}

%-----------------------------------------------------------------------
%                     Author Index Specification
%-----------------------------------------------------------------------
% Specify how each author name should appear in the author index.  The 
% \paindex{ } should be used to indicate the primary author, and the
% \aindex for all other co-authors.  You MUST use the following
% syntax: 
%
% SYNTAX:  \aindex{Lastname, F. M.}
% 
% where F is the first initial and M is the second initial (if
% used).  This guarantees that authors that appear in multiple papers
% will appear only once in the author index.  
%
% EXAMPLE: \paindex{Crabtree, D.}
%          \aindex{Manset, N.}        
%          \aindex{Veillet, C.}        
%
% NOTE: this information is also used to build the author list that
% appears in the table of contents.  Authors will be listed in the order
% of the \paindex and \aindex commmands.
%

\paindex{Pety, J.}
\aindex{Gueth, F.}     % Remove this line if there is only one author
\aindex{Guilloteau, S.}
\aindex{Lucas, S.}
\aindex{Teuben, P. J.}
\aindex{Wright, M. C. H.}

%-----------------------------------------------------------------------
%                     Author list for page header       
%-----------------------------------------------------------------------
% Please supply a list of author last names for the page header. in
% one of these formats:
%
% EXAMPLES:
% \authormark{Lastname}
% \authormark{Lastname1 \& Lastname2}
% \authormark{Lastname1, Lastname2, ... \& LastnameN}
% \authormark{Lastname et al.}
%
% Use the "et al." form in the case of seven or more authors, or if
% the preferred form is too long to fit in the header.

\authormark{Pety, Gueth, Guilloteau, Lucas, Teuben \& Wright}

%-----------------------------------------------------------------------
%                       Subject Index keywords
%-----------------------------------------------------------------------
% Enter a comma separated list of up to 6 keywords describing your
% paper.  These will NOT be printed as part of your paper; however,
% they will be used to generate the subject index for the proceedings.
% There is no standard list; however, you can consult the indices
% for past proceedings (http://adass.org/adass/proceedings/).
%
% EXAMPLE:  \keywords{visualization, astronomy: radio, parallel
%                     computing, AIPS++, Galactic Center}
%
% In this example, the author noticed that "radio astronomy" appeared
% in the ADASS VII Index as "astronomy" being the major keyword and
% "radio" as the minor keyword.  The colon is used to introduce another
% level into the index.

\keywords{software: GILDAS, software: MIRIAD, astronomy: radio,
  interoperability, python}

%-----------------------------------------------------------------------
%                              Abstract
%-----------------------------------------------------------------------
% Type abstract in the space below.  Consult the User Guide and Latex
% Information file for a list of supported macros (e.g. for typesetting 
% special symbols). Do not leave a blank line between \begin{abstract} 
% and the start of your text.

\begin{abstract}          % Leave intact
  We describe here the first step we took to interoperate \GILDAS{} and
  \MIRIAD{}, two current state--of--the--art data reduction packages of
  millimeter radioastronomy.
\end{abstract}

%-----------------------------------------------------------------------
%                             Main Body
%-----------------------------------------------------------------------
% Place the text for the main body of the paper here.  You should use
% the \section command to label the various sections; use of
% \subsection is optional.  Significant words in section titles should
% be capitalized.  Sections and subsections will be numbered
% automatically. 
%
% EXAMPLE:  \section{Introduction}
%           ...
%           \subsection{Our View of the World}
%           ...
%           \section{A New Approach}
%
% It is recommended that you look at the sample papers, sample1.tex
% and sample2.tex, for examples for formatting references, footnotes,
% figures, equations, html links, lists, and other special features.  

\section{Introduction to \GILDAS{} and \MIRIAD{} packages}

\begin{figure}[t]
  \epsscale{0.85}
  \plotone{P4-14_f1.eps}
  \caption{Results of the evaluation of the \GILDAS{} and \MIRIAD{}
    packages for compliance with the \ALMA{} off-line data processing
    requirements. Those requirements have been prioritized: \texttt{All}
    takes into account absolutely all the requirements while
    \texttt{Critical} only evaluate the 49\% more important requirements.
    In the legend, A means Adequate, A/E Enhancements are desired, I/N
    Inadequate of Not available and U Unable to evaluate.  Low/Medium/High
    is the severity of the failure to meet the requirement.  Many more
    details may be found in Pety et al.  (2003a).}
  \label{fig:P4-14_f1}
\end{figure}

\GILDAS{} and \MIRIAD{} (Sault et al. 1994) are two state-of-the-art data
reduction packages for the current generation of millimeter instruments.
\GILDAS{} is \emph{daily} used at \IRAM{} instruments (\ie{} the Plateau de
Bure Interferometer and the 30~m) as well as several other single dish
telescopes (\eg{} \CSO{}, HHT, Effelsberg) while \MIRIAD{} is \emph{daily}
used at BIMA, ATCA, OVRO and WSRT. Moreover, as revealed by publicly
available evaluations for compliance with the \ALMA{} off-line data
processing requirements (Gueth et al. 2003 and Wright et al. 2003), almost
2/3 of the core functionalities needed by the \emph{next} generation of mm
interferometers are adequately covered by \GILDAS{} and \MIRIAD{}.  Going
further, Fig.~\ref{fig:P4-14_f1} shows that, between them, \GILDAS{} and
\MIRIAD{} cover more that 75\% of the \ALMA{} requirements for off-line.
Indeed, although the core functionalities needed to reduce millimeter
interferometry data are well covered in both packages, the strength of both
packages are complementary.  \GILDAS{} has good data analysis and
visualization tools, while \MIRIAD{} has a complete set of calibration and
imaging algorithms including polarization (for more details, see Pety et
al. 2003a). It is thus interesting to study the possibility of
interoperating both packages in a user-friendly way (Pety et al. 2003b).

Both packages have rather different implementation philosophy. \GILDAS{} is
made of large stand-alone programs like \CLASS{} and \CLIC{} for
single-dish and interferometric data calibration and \MAPPING{} for data
imaging, deconvolution and analysis. In addition, \GILDAS{} has the ability
to launch fully separated tasks.  \MIRIAD{} is build around a large
collection of individual, well focused tasks, that run under the native
operating system.  Both packages also have different user interfaces.
\GILDAS{} is build around \SIC{}, an homemade command line interface
enabling direct insight into the data, and \GREG{}, a very efficient,
homemade plotting environment.  Since \MIRIAD{} programs are simple
standalone tasks, every UNIX shell may serve as \MIRIAD{} user interface.
\MIRIAD{} developers have in the past experiment with graphical user
interfaces, but in practice users write shell scripts to perform their
analysis pipelines.  Finally, although the data models used in \GILDAS{}
and \MIRIAD{} are very similar, their implementation is very different.

In a first step toward interoperability, we thus decided to run both
packages under \python{} and to exchange data using FITS.  The porting of both
packages under \python{} is described in the next section. Section~3 describes
a concrete example of \GILDAS{}/\MIRIAD{} interoperability. This gives us
the opportunity to compare the imaging and deconvolution algorithms of
\GILDAS{} and \MIRIAD{}.

\section{Python Ports}

The main difficulty to port \GILDAS{} under \python{} is the different
characteristics of \SIC{} vs.\ \python{}, \ie{} strongly vs.\ weakly typed
languages and procedural vs.\ object oriented languages. As a first
approach, we used \fpig{} (a \FORTRAN{} to \python{} interface generator)
to produce a generic interface to the \SIC{} command line
interpretor and a direct interface to some of the most commonly used
commands. All \GILDAS{} capabilities are thus directly available from the
\python{} prompt. No one--to--one correspondence between \FORTRAN{} and
\python{} variables is yet provided impairing the good interaction with
data. Progress is currently being made in this direction. Another foreseen
evolution is the possibility to swap on--the--fly from \python{} to \SIC{}
and vice-versa, as this is an important requirement for the community of
current users.

In the \MIRIAD{} case, a generic \python{} interface was written to pass
information to \MIRIAD{} tasks and recover their output.  No direct access
to \MIRIAD{} internal data structure or automatic translation of template
procedures from \texttt{csh} to \python{} is yet available. The next
improvement will be to automate the building of task access in \python{},
decreasing the size of current \python{} script by about 50\%.  Finally,
the ATA group have recently implemented a \texttt{Jython} interface to
\MIRIAD{} dubbed \texttt{J-MIRIAD} (Harp \& Wright 2003).

\section{Interoperability Demonstration}

\begin{figure}[t]
  \epsscale{0.85}
  \plotone{P4-14_f2.eps}
  \caption{Top images: The 1~mm continuum imaged in \GILDAS{} (right) and
    \MIRIAD{} (left), both using the same map resolution and restoring
    beam.  Bottom left: $|$\GILDAS{}$-$\MIRIAD{}$|$. Bottom right:
    \GILDAS{}/$|$\GILDAS{}$-$\MIRIAD{}$|$. In all panels except the
    fidelity image, the same color scale and 3--$\sigma$ contours levels
    have been used. The difference map sigma is $1.27\times10^{-3}$ compared to
    $1.76\times10^{-3}$ for \GILDAS{} and \MIRIAD{} images. Please, note the
    absence of contours on the difference image and the high fidelities
    similar as the image dynamic, which imply an excellent agreement
    between \GILDAS{} and \MIRIAD{}.}
  \label{fig:P4-14_f2}
\end{figure}

An example of the current interoperability capabilities of \GILDAS{} and
\MIRIAD{} is available with all needed details to rerun it on your own
PC\footnote{The URL is:
  \texttt{http://www.iram.fr/\~{}gildas/gildas-miriad/gildas-miriad.html}.}.
In this example, we start from the raw visibilities (in native \PdBI{}
format) of the Young Stellar Object GG~Tau (Guilloteau, Dutrey \& Simon
1999). The data is calibrated inside \GILDAS{}, imaged and deconvolved
inside \MIRIAD{} and finally visualized back inside \GILDAS{}, both
packages being called from the \emph{same \python{} process}. This is about
1.5~GB of data that has been processed in about 10 minutes end--to--end on
a 1.5~GHz AMD Athlon with 768~MB of RAM memory. We had to adapt our FITS
readers/writers as uvfit is an ill-defined standard that does not fully
support the flexible \GILDAS{} and \MIRIAD{} data structures. No others
difficulties were encountered. The main results are velocity channel maps
of the \HCOp{} \Jone{} line (not shown here) and a 1mm continuum map.

As a by-product, we also imaged and deconvolved the 1mm continuum map
inside \GILDAS{}. Fig.~\ref{fig:P4-14_f2} shows a detailed comparison of the
\GILDAS{} and \MIRIAD{} images, which allows us to conclude that \GILDAS{}
and \MIRIAD{} imaging and deconvolution performances are similar.

%-----------------------------------------------------------------------
%                             References
%-----------------------------------------------------------------------
% List your references below within the reference environment
% (i.e. between the \begin{references} and \end{references} tags).
% Each new reference should begin with a \reference command which sets
% up the proper indentation.  Observe the following order when listing
% bibliographical information for each reference:  author name(s),
% publication year, journal name, volume, and page number for
% articles.  Note that many journal names are available as macros; see
% the User Guide listing "macro-ized" journals.   
%
% EXAMPLE:  \reference Hagiwara, K., \& Zeppenfeld, D.\  1986, 
%                Nucl.Phys., 274, 1
%           \reference H\'enon, M.\  1961, Ann.d'Ap., 24, 369
%           \reference King, I.\ R.\  1966, \aj, 71, 276
%           \reference King, I.\ R.\  1975, in Dynamics of Stellar 
%                Systems, ed.\ A.\ Hayli (Dordrecht: Reidel), 99
%           \reference Tody, D.\  1998, \adassvii, 146
%           \reference Zacharias, N.\ \& Zacharias, M.\ 2003,
%                \adassxii, \paperref{P7.6}
% 
% Note the following tricks used in the example above:
%
%   o  \& is used to format an ampersand symbol (&).
%   o  \'e puts an accent agu over the letter e.  See the User Guide
%      and the sample files for details on formatting special
%      characters.  
%   o  "\ " after a period prevents LaTeX from interpreting the period 
%      as an end of a sentence.
%   o  \aj is a macro that expands to "Astron. J."  See the User Guide
%      for a full list of journal macros
%   o  \adassvii is a macro that expands to the full title, editor,
%      and publishing information for the ADASS VII conference
%      proceedings.  Such macros are defined for ADASS conferences I
%      through XI.
%   o  When referencing a paper in the current volume, use the
%      \adassxii and \paperref macros.  The argument to \paperref is
%      the paper ID code for the paper you are referencing.  See the 
%      note in the "Paper ID Code" section above for details on how to 
%      determine the paper ID code for the paper you reference.  
%
\begin{references}
\reference Gueth, F., Guilloteau, S., Lucas, R., Pety, J., \&
    Wright, M.~C.~H., 2003, BIMA Memo 96 \& IRAM Memo 2003-2.
\reference Guilloteau, S., Dutrey, A. \& Simon, M., 1999, \aap, 348, 570.
\reference Harp, G.~R. \& Wright, M.~C.~H., 2003, BIMA Memo 98 \& 
    ATA Memo 58.
\reference Pety, J., Gueth, F., Guilloteau, S., Teuben, P.~J. \&
    Wright, M.~C.~H., 2003, \ALMA{} Memo 464.
\reference Pety, J., Gueth, F., Guilloteau, S., Teuben, P.~J. \&
    Wright, M.~C.~H., 2003, \ALMA{} Memo 465.
\reference Sault, R.~J., Teuben, P.~J. \& Wright, M.~C.~H., 1994, 
    \adassiv, 433
\reference Wright, M.~C.~H., Teuben, P.~J., \& Pety, J., 2003,
    BIMA Memo 95 \& IRAM Memo 2003-1.
\end{references}

% Do not place any material after the references section

\end{document}  % Leave intact
