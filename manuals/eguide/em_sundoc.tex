\module{mir.bug.csh}%
\noindent Send mail about bugs to appropriate person
\newline \ 
\newline \abox{File:} \$MIR/src/scripts/sun/mir.bug.csh
\newline \abox{Responsible:} Bart Wakker
\newline \abox{Keywords:} user utility
\newline{\tenpoint\newline
Usage: mir.bug.csh [selection] [taskname] [file]

Mirbug allows the user to send mail to the appropriate persons.

[selection] should be one of ``bug'' or ``feedback''. This influences
the information prepended to a report file. ``mirbug'' is an alias
for ``mir.bug.csh bug''; `` mirfeedback'' is an alias for ``mir.bugcsh.
feedback''

[taskname] should be one of the miriad tasks/tools/scripts/subs.
It is used to find out who is responsible for the code and then the
report is mailed to that person. If it is not given mirbug complains,
but continues.

[file] is the name of an optional, previously created, file
containing the report. Some extra information is prepended and an
editor is started. The default editor is vi on unix and edt on VMS.
However, by including 'setenv EDITOR mem' in the .cshrc file (unix)
or 'editor:==mem' in the login.com file (VMS) it is possible to
specify to use, for instance, mem (or any other editor).

[file] must be the third argument.

If mir.bug.csh is aborted and a template was already generated, it is
saved to allow extra editing, otherwise it is deleted.

The generated mail message is sent to the responsible person and to
a central address at the site of the programmer. If no responsible
person can be found it is sent to all the central addresses. A copy
is also sentto the user. 

Alias:  ``mirbug'' is an alias for ``mir.bug.csh bug''
Alias:  ``mirfeedback'' is an alias for ``mir.bug.csh feedback''
\par}
\module{mir.filetar}%
\noindent Make a tarfile of selected MIRIAD files
\newline \ 
\newline \abox{File:} \$MIR/src/scripts/sun/mir.filetar
\newline \abox{Responsible:} Mark Stupar
\newline \abox{Keywords:} system operation
\newline{\tenpoint\newline
Make a tarfile of selected MIRIAD files.  The tarfile
will have the name ``ddmmmyy.hostname'', and is intended
for transmission to other sites for source code
updating.  The script MUST be run from the \$MIR
directory.

{\eightpoint\begintt
  Usage:  mir.filetar $1 $2

  $1 ... directory in which to put the tarfile
         (typically, this is $MIR/adm/export)
  $2 ... name of the file containing MIRIAD filenames

\endtt}
All files in \$2 must reference MIRIAD files with respect
to \$MIR.  Example:  to make a tarfile of MIRIAD sourcecode
avsub.for and fits.for, \$2 should have the entries

{\eightpoint\begintt
  ./src/prog/analysis/avsub.for
  ./src/prog/convert/fits.for

\endtt}
Caution:  due to the mandatory name structure, like-named
files already existing in the output directory will be
overwritten if the script is run more than once in a day.
\par}
\module{mir.find}%
\noindent Search for a file in MIRIAD source code
\newline \ 
\newline \abox{File:} \$MIR/src/scripts/sun/mir.find
\newline \abox{Responsible:} Bart Wakker
\newline \abox{Keywords:} programmer tool
\newline{\tenpoint\newline
Within \$MIRSRC directory, find any file or directory 
containing characters \$1.

{\eightpoint\begintt
  Usage:  mir.find  $1

    $1 ... <character string>

\endtt}
Alias:  ``mirfind'' is an alias for ``mir.find''.
\par}
\module{mir.gendoc}%
\noindent Generate on-line MIRIAD documentation
\newline \ 
\newline \abox{File:} \$MIR/src/scripts/sun/mir.gendoc
\newline \abox{Responsible:} Mark Stupar
\newline \abox{Keywords:} system operation
\newline{\tenpoint\newline
Load MIRIAD on-line documentation files.

Remove all files in \$MIRPDOC and \$MIRSDOC; install fresh
copies of *.doc files (tasks) and *.cdoc files (scripts) into
\$MIRPDOC; install fresh copies of *.sdoc files (subroutines)
*.tdoc (subs internal to programs) into \$MIRSDOC; remove
indices and category listings from \$MIR/doc/misc and replace
them with fresh copies.

{\eightpoint\begintt
  Usage:  mir.gendoc
\endtt}
The script is intended for routine maintenance, to be done
after source code updates.
\par}
\module{mir.hcconv}%
\noindent Load MIRIAD program HCCONV from source to \$MIRBIN
\newline \ 
\newline \abox{File:} \$MIR/src/scripts/sun/mir.hcconv
\newline \abox{Responsible:} Mark Stupar
\newline \abox{Keywords:} system operation
\newline{\tenpoint\newline
Load MIRIAD program HCCONV from source code to \$MIRBIN.

{\eightpoint\begintt
  Usage:  mir.hcconv
\endtt}
\par}
\module{mir.hdf}%
\noindent Compile HDF source code into \$MIRLIB/libdf.a
\newline \ 
\newline \abox{File:} \$MIR/src/scripts/sun/mir.hdf
\newline \abox{Responsible:} Mark Stupar
\newline \abox{Keywords:} system operation
\newline{\tenpoint\newline
Compile HDF source code into \$MIRLIB/libdf.a for use with
MIRIAD program MXDS.

{\eightpoint\begintt
  Usage:  mir.hdf $argv

  $argv = name of HDF source code files within $MIR/borrow/hdf
          to be compiled into the library; if there are no
          arguments, all source code will be compiled into
          $MIRLIB/libdf.a.  Wildcards are not allowed when
          specifying HDF source code filenames.
\endtt}
\par}
\module{mir.help}%
\noindent Obtain programmer-useful help about MIRIAD source files.
\newline \ 
\newline \abox{File:} \$MIR/src/scripts/sun/mir.help
\newline \abox{Responsible:} Bart Wakker
\newline \abox{Keywords:} programmer tool
\newline{\tenpoint\newline
Find information about all tasks/subroutines whose name matches all
or part of the input.

This is a general purpose help-script. Some other scripts actually
just call this one in a particular way:
{\eightpoint\begintt
   mir.pindex is equivalent to 'mir.help tasks -i'
   mir.sindex is equivalent to 'mir.help subroutines -i'
   mir.find   is equivalent to 'mir.help "!*" -n'

\endtt}
Usage:  mir.help [topic or routine] [option] [name]

topic may be:
{\eightpoint\begintt
       'tasks' ......... generate a list of tasks below $MIRPROG.
       'subroutines' ... generate a list of the subroutines.
\endtt}
routine may be:
{\eightpoint\begintt
       <any MIRIAD task or subroutine> ... generate the formatted
                                           on-line doc for the task
                                           or subroutine.

\endtt}
option may be:
{\eightpoint\begintt
       -s ... Subroutine sources have search precedence over
              existing pre-formatted files.
       -n ... Search for the full name of the specified file.
       -i ... run DOC to create an index for the given topic.
       -t ... run DOC to create an index by category for the given
              topic.

\endtt}
[name] allows to select one routine if option was -i or -t.
Alias:  ``mirhelp'' is an alias for ``mir.help''.
\par}
\module{mir.linpack}%
\noindent Compile linpack source into \$MIRLIB/liblinpack.a
\newline \ 
\newline \abox{File:} \$MIR/src/scripts/sun/mir.linpack
\newline \abox{Responsible:} Mark Stupar
\newline \abox{Keywords:} system operation
\newline{\tenpoint\newline
Compile linpack source code into \$MIRLIB/liblinpack.a for use
with any MIRIAD task or subroutine.

{\eightpoint\begintt
  Usage:  mir.linpack $argv

  $argv = name of linpack source code files within 
          $MIR/borrow/linpack to be compiled into the library;
          if there are no arguments, all source code will be
          compiled into $MIRLIB/liblinpack.a.  Wildcards are
          not allowed when specifying source code filenames.
\endtt}
\par}
\module{mir.loadall}%
\noindent Load MIRIAD subsystems from source code
\newline \ 
\newline \abox{File:} \$MIR/src/scripts/sun/mir.loadall
\newline \abox{Responsible:} Mark Stupar
\newline \abox{Keywords:} system operation
\newline{\tenpoint\newline
Load the MIRIAD system (or specified parts of it) from source
code and write log files to \$MIR/tmp.  The script is intended
to be run in background.

{\eightpoint\begintt
  Usage:  mir.loadall $argv

  $argv = none means to load the entire system from source code;
          the order is scripts, linpack, pgfont, pgplot, subs,
          hdf, tools, uvdisplay, hcconv, uvhat, msss, prog, mxds,
          gendoc.

          Specifying one or more of these will load only the
          specified section in the order given by the user (only
          those parts given above are recognized).  Example:
          ``mir.loadall subs prog &'' will compile all of the subs
          into $MIRLIB/libmir.a and then load all of the tasks
          within $MIRSRC/prog into $MIRBIN, leaving a log of the
          work in directory $MIR/tmp.
\endtt}
\par}
\module{mir.mkdirs}%
\noindent Make the directories for a MIRIAD system
\newline \ 
\newline \abox{File:} \$MIR/src/scripts/sun/mir.mkdirs
\newline \abox{Responsible:} Mark Stupar
\newline \abox{Keywords:} system operation
\newline{\tenpoint\newline
Make the directories for a MIRIAD system.  If some/any
directories already exist, they will not be erased,
and needed directories that don't already exist will
be created.  However, all directories will be set with
permissions 755, except the adm directory and its
subdiretories, which have permissions 775.

{\eightpoint\begintt
  Usage:  mir.mkdirs $1

  $1 ... top-level directory (effectively, the $MIR
         directory for this structure)
\endtt}
\par}
\module{mir.mkdirs1}%
\noindent Internal subscript for script mir.mkdirs
\newline \ 
\newline \abox{File:} \$MIR/src/scripts/sun/mir.mkdirs1
\newline \abox{Responsible:} Mark Stupar
\newline \abox{Keywords:} system operation
\newline{\tenpoint\newline
Internal subscript for script mir.mkdirs.

{\eightpoint\begintt
  Usage:  none ... the subscript is used only by script
                   mir.mkdirs1
\endtt}
\par}
\module{mir.msss}%
\noindent Load MIRIAD's version of the AIPS' SSS
\newline \ 
\newline \abox{File:} \$MIR/src/scripts/sun/mir.msss
\newline \abox{Responsible:} Mark Stupar
\newline \abox{Keywords:} system operation
\newline{\tenpoint\newline
Load MIRIAD program MXDS from source code to \$MIRBIN.

{\eightpoint\begintt
  Usage:  mir.msss
\endtt}
\par}
\module{mir.mxds}%
\noindent Load MIRIAD's version of NCSA's XDS
\newline \ 
\newline \abox{File:} \$MIR/src/scripts/sun/mir.mxds
\newline \abox{Responsible:} Mark Stupar
\newline \abox{Keywords:} system operation
\newline{\tenpoint\newline
Load MIRIAD program MXDS from source code to \$MIRBIN.

{\eightpoint\begintt
  Usage:  mir.mxds
\endtt}
\par}
\module{mir.pgfont}%
\noindent Load PGPLOT fonts from source code
\newline \ 
\newline \abox{File:} \$MIR/src/scripts/sun/mir.pgfont
\newline \abox{Responsible:} Mark Stupar
\newline \abox{Keywords:} system operation
\newline{\tenpoint\newline

Load \$MIRLIB/grfont.dat (the PGPLOT fonts) from source
code.

{\eightpoint\begintt
  Usage:  mir.pgfont

\endtt}

\par}
\module{mir.pgplot}%
\noindent Load the PGPLOT library from source code
\newline \ 
\newline \abox{File:} \$MIR/src/scripts/sun/mir.pgplot
\newline \abox{Responsible:} Mark Stupar
\newline \abox{Keywords:} system operation
\newline{\tenpoint\newline
Load \$MIRLIB/libpgplot.a from pgplot source code.

{\eightpoint\begintt
  Usage:  mir.pgplot $argv

  $argv = name of PGPLOT source code files within directory
          $MIR/borrow/pgplot to be compiled into the library;
          if there are no arguments, all source code will
          be compiled into $MIRLIB/libpgplot.a.  Wildcards
          are not allowed when specifying PGPLOT source code
          filenames.
\endtt}
\par}
\module{mir.pindex}%
\noindent Make alphabetic index of MIRIAD tasks
\newline \ 
\newline \abox{File:} \$MIR/src/scripts/sun/mir.pindex
\newline \abox{Responsible:} Bart Wakker
\newline \abox{Keywords:} programmer tool
\newline{\tenpoint\newline
Make an alphabetic index of all MIRIAD tasks within
directory \$MIRSRC/prog whose name or one-line description
contains the string 'string'.  ``mir.pindex \$1'' is
equivalent to ``mir.help tasks -i \$1''.

{\eightpoint\begintt
  Usage:  mir.pindex [string]

  string ... the full/partial filename to be found

\endtt}
Alias:  ``mirindex'' is an alias for ``mir.pindex''.
\par}
\module{mir.prog}%
\noindent Load MIRIAD tasks from source code to \$MIRBIN
\newline \ 
\newline \abox{File:} \$MIR/src/scripts/sun/mir.prog
\newline \abox{Responsible:} Mark Stupar
\newline \abox{Keywords:} system operation
\newline{\tenpoint\newline
Load MIRIAD tasks in directory \$MIRSRC/prog from source
code into \$MIRBIN, using the MIRIAD libraries in \$MIRLIB
to resolve external references.  All FORTRAN source code
has extension *.for and is processed by the RATTY (MIRIAD's
FORTRAN preprocessor) program.

{\eightpoint\begintt
  Usage:  mir.prog $argv

  $argv = name of source code file(s) within the home
          directory of the source code (for example,
          the source code for task ZEEFAKE is zeefake.for,
          and resides in $MIR/src/prog/analysis; entering
          ``mir.prog zeefake.for'' compiles and loads the
          program).  If there are no arguments, all source
          code within directory $MIR/src/prog will be
          compiled and loaded into $MIRBIN.  Wildcards are
          not allowed when specifying source code filenames.
\endtt}
\par}
\module{mir.prog1}%
\noindent Internal subscript for script mir.prog
\newline \ 
\newline \abox{File:} \$MIR/src/scripts/sun/mir.prog1
\newline \abox{Responsible:} Mark Stupar
\newline \abox{Keywords:} system operation
\newline{\tenpoint\newline
The subscript loads MIRIAD tasks in directory
\$MIR/src/prog into \$MIRBIN.

{\eightpoint\begintt
  Usage:  none ... the subscript is used only by script
          mir.prog.
\endtt}
\par}
\module{mir.scripts}%
\noindent Copy scripts from source to \$MIRBIN
\newline \ 
\newline \abox{File:} \$MIR/src/scripts/sun/mir.scripts
\newline \abox{Responsible:} Mark Stupar
\newline \abox{Keywords:} system operation
\newline{\tenpoint\newline
All files in \$MIR/src/scripts/sun beginning with the characters
``mir.'' are copied to \$MIRBIN (if there are no arguments; else
the specified files are copied), after first backing off those
already in \$MIRBIN to \$MIR/oldsrc.

{\eightpoint\begintt
  Usage:  mir.scripts $argv

  $argv = name of source code file(s) within directory
          $MIR/src/scripts/sun.  If there are no arguments,
          all source code within the directory will be copied
          into $MIRBIN.  Wildcards are not allowed when
          specifying source code filenames.
\endtt}
\par}
\module{mir.setpath.awk}%
\noindent Awk script to insert \$MIRBIN in searchpath
\newline \ 
\newline \abox{File:} \$MIR/src/scripts/sun/mir.setpath.awk
\newline \abox{Responsible:} Mark Stupar
\newline \abox{Keywords:} system operation
\newline{\tenpoint\newline
The script places \$MIRBIN in the user's directory searchpath
immediately after the ``.'' directory (if that is in the path),
and first otherwise.

{\eightpoint\begintt
  Usage:  none ... the script is sourced by $MIR/MIRRC.

\endtt}
Beware:  the script assumes that the path is passed in the
form generated by the \$MIR/MIRRC file (ie, first) and then
parses the path to determine whether directory ``.'' is
present.
\par}
\module{mir.sindex}%
\noindent Make alphabetic index of MIRIAD subroutines
\newline \ 
\newline \abox{File:} \$MIR/src/scripts/sun/mir.sindex
\newline \abox{Keywords:} programmer tool
\newline \abox{Responsible:} Bart Wakker
\newline{\tenpoint\newline
Make an alphabetic index of all MIRIAD subroutines in
directory \$MIRBIN whose name or one-line description
contains the string 'string'.  ``mir.sindex string'' is
equivalent to ``mir.help subroutines -i string''.

{\eightpoint\begintt
  Usage:  mir.pindex [string]

  string ... the string
\endtt}
\par}
\module{mir.submit}%
\noindent Programmer submission of code to MIRIAD
\newline \ 
\newline \abox{File:} \$MIR/src/scripts/sun/mir.submit
\newline \abox{Responsible:} Peter Teuben
\newline \abox{Keywords:} system operation
\newline{\tenpoint\newline
Submit source code from programmers to the MIRIAD system.
Using a lockfile, source code permissions are changed after
code checkout by programmers to prevent overwriting by
another programmer.

{\eightpoint\begintt
  Usage:  mir.submit [-m] [-i] [-p] source destdir

  -m ........ move files rather than copying
  -i ........ run non-interactively
  -p ........ preserved timestamp when copying
  source .... source code files(s)
  destdir ... destination directory

\endtt}
The script requires minor system modifications and is used
only by the Maryland MIRIAD programmers.

Alias:  ``mirsubmit'' is an alias for ``mir.submit''.
\par}
\module{mir.subs}%
\noindent Compile MIRIAD subroutines into \$MIRLIB/libmir.a
\newline \ 
\newline \abox{File:} \$MIR/src/scripts/sun/mir.subs
\newline \abox{Responsible:} Mark Stupar
\newline \abox{Keywords:} system operation
\newline{\tenpoint\newline
Compile MIRIAD subroutines in directory \$MIRSUBS from source
code into \$MIRLIB/libmir.a.  All FORTRAN source code has
extension *.for and is processed by the RATTY (MIRIAD's
FORTRAN preprocessor) program.

{\eightpoint\begintt
  Usage:  mir.subs $argv

  $argv = name of source code file(s) within directory 
          $MIRSUBS.  If there are no arguments, all source
          code within directory $MIRSUBS will be compiled
          into $MIRLIB/libmir.a.  Wildcards are not allowed
          when specifying source code filenames.
\endtt}
\par}
\module{mir.systar}%
\noindent Make a tarfile of MIRIAD source files.
\newline \ 
\newline \abox{File:} \$MIR/src/scripts/sun/mir.systar
\newline \abox{Responsible:} Mark Stupar
\newline \abox{Keywords:} system operation
\newline{\tenpoint\newline
Make a tarfile of the transmittable MIRIAD system
(everything in directories \$MIR/borrow, \$MIR/cat, 
\$MIR/manuals, and \$MIR/src).

{\eightpoint\begintt
  Usage:  mir.systar $1

  $1 ... the name of the output tarfile
\endtt}
\par}
\module{mir.tar2sys}%
\noindent Update system from a MIRIAD tarfile
\newline \ 
\newline \abox{File:} \$MIR/src/scripts/sun/mir.tar2sys
\newline \abox{Responsible:} Mark Stupar
\newline \abox{Keywords:} system operation
\newline{\tenpoint\newline
Update the system from a MIRIAD tarfile, saving whatever
is to be overwritten to \$MIR/oldsrc and appending '.ddmmyy' to
each file saved.  Timestamps are preserved in the saved
files.  Tarfile names must be given with respect to \$MIR,
and the contents of the tarfile must be give with respect
to \$MIR, since MIRIAD tarfiles are always created from
directory \$MIR.

Sample tarfile contents:

{\eightpoint\begintt

  ./src/prog/analysis/boxspec.for
  ./src/subs/shadowed.for
  ./cat/keywords

  Usage:  mir.tar2sys $1

  $1 ... the name of the MIRIAD tarfile (typically, a
         file in $MIR/adm/import)
\endtt}
\par}
\module{mir.tools}%
\noindent Load MIRIAD's tools from source code into \$MIRBIN
\newline \ 
\newline \abox{File:} \$MIR/src/scripts/sun/mir.tools
\newline \abox{Responsible:} Mark Stupar
\newline \abox{Keywords:} system operation
\newline{\tenpoint\newline
Load MIRIAD's utility programs in directory \$MIRSRC/tools
from source code into \$MIRBIN.  These include programs
DOC, FLINT, MIRIAD, MIRTOOL, PANEL, and RATTY.

{\eightpoint\begintt
  Usage:  mir.tools $argv

  $argv = name of source code file(s) within directory
          $MIRSRC/tools.  If there are no arguments, all
          source code within directory $MIRSRC/tools will be
          compiled and loaded into $MIRBIN.  Wildcards are
          not allowed when specifying source code filenames.
\endtt}
\par}
\module{mir.uvdisplay}%
\noindent Load UVDISPLAY from source code to \$MIRBIN
\newline \ 
\newline \abox{File:} \$MIR/src/scripts/sun/mir.uvdisplay
\newline \abox{Responsible:} Mark Stupar
\newline \abox{Keywords:} system operation
\newline{\tenpoint\newline
Load MIRIAD program UVDISPLAY from source code to \$MIRBIN.

{\eightpoint\begintt
  Usage:  mir.uvdisplay
\endtt}
\par}
\module{mir.uvhat}%
\noindent Load UNIX UVHAT from source code to \$MIRBIN
\newline \ 
\newline \abox{File:} \$MIR/src/scripts/sun/mir.uvhat
\newline \abox{Responsible:} Mark Stupar
\newline \abox{Keywords:} system operation
\newline{\tenpoint\newline
Load MIRIAD's (UNIX) UVHAT from source code to \$MIRBIN.

{\eightpoint\begintt
  Usage:  mir.uvhat
\endtt}
\par}
\module{mirbug}%
\noindent Send mail about bugs to appropriate person
\newline \ 
\newline \abox{File:} \$MIR/src/scripts/sun/mir.bug.csh
\newline \abox{Responsible:} Bart Wakker
\newline \abox{Keywords:} user utility
\newline{\tenpoint\newline
Usage: mirbug [taskname] [file]

Mirbug allows the user to send mail to the appropriate persons.

[taskname] should be one of the miriad tasks/tools/scripts/subs.
It is used to find out who is responsible for the code and then the
report is mailed to that person. If it is not given mirbug complains,
but continues.

[file] is the name of an optional, previously created, file
containing the report. Some extra information is prepended and an
editor is started. The default editor is vi on unix and edt on VMS.
However, by including 'setenv EDITOR mem' in the .cshrc file (unix)
or 'editor:==mem' in the login.com file (VMS) it is possible to
specify to use, for instance, mem (or any other editor).

[file] must be the third argument.

If mir.bug.csh is aborted and a template was already generated, it is
saved to allow extra editing, otherwise it is deleted.

The generated mail message is sent to the responsible person and to
a central address at the site of the programmer. If no responsible
person can be found it is sent to all the central addresses. A copy
is also sentto the user. 

Alias:  ``mirbug'' is an alias for ``mir.bug.csh bug''.
\par}
\module{mirfeedback}%
\noindent Send feedback to the appropriate person
\newline \ 
\newline \abox{File:} \$MIR/src/scripts/sun/mir.bug.csh
\newline \abox{Responsible:} Bart Wakker
\newline \abox{Keywords:} user utility
\newline{\tenpoint\newline
Usage: mirfeedback [taskname] [file]

This is a way to send feedback to a programmer. For a description
of how it works, see ``mirbug''. The difference between these two is
in the prepended information.

Alias:  ``mirfeedback'' is an alias for ``mir.bug.csh feedback''.
\par}
\module{mirfind}%
\noindent Search for a file in MIRIAD source code
\newline \ 
\newline \abox{File:} \$MIR/src/scripts/sun/mir.find
\newline \abox{Responsible:} Bart Wakker
\newline \abox{Keywords:} programmer tool
\newline{\tenpoint\newline
Within \$MIRSRC directory, find any file or directory 
containing characters \$1.

{\eightpoint\begintt
  Usage:  mirfind  $1

    $1 ... <character string>

\endtt}
Alias:  ``mirfind'' is an alias for ``mir.find''.
\par}
\module{mirhelp}%
\noindent Obtain programmer-useful help about MIRIAD source files.
\newline \ 
\newline \abox{File:} \$MIR/src/scripts/sun/mir.help
\newline \abox{Responsible:} Bart Wakker
\newline \abox{Keywords:} programmer tool
\newline{\tenpoint\newline
Find information about all tasks/subroutines whose name matches all
or part of the input.

Usage:  mirhelp [topic or routine] [option] [name]

topic may be:
{\eightpoint\begintt
       'tasks' ......... generate a list of tasks below $MIRPROG.
       'subroutines' ... generate a list of the subroutines.
\endtt}
routine may be:
{\eightpoint\begintt
       <any MIRIAD task or subroutine> ... generate the formatted
                                           on-line doc for the task
                                           or subroutine.

\endtt}
option may be:
{\eightpoint\begintt
       -s ... Subroutine sources have search precedence over
              existing pre-formatted files.
       -n ... Search for the full name of the specified file.
       -i ... run DOC to create an index for the given topic.
       -t ... run DOC to create an index by category for the given
              topic.

\endtt}
name allows to select one routine if option was -i or -t.

Alias:  ``mirhelp'' is an alias for ``mir.help''.
\par}
\module{mirindex}%
\noindent Give one-line description of MIRIAD programs
\newline \ 
\newline \abox{File:} \$MIR/src/scripts/sun/mir.pindex
\newline \abox{Responsible:} Bart Wakker
\newline \abox{Keywords:} user utility
\newline{\tenpoint\newline
Make an alphabetic index of all MIRIAD tasks within
directory \$MIRSRC/prog whose name or one-line description
contains the string 'string'.  ``mirindex \$1'' is
equivalent to ``mir.help tasks -i \$1''.

{\eightpoint\begintt
  Usage:  mirindex [string]

  Alias:  ``mirindex'' is an alias for ``mir.pindex''.
\endtt}
\par}
\module{MIRRCsun3}%
\noindent MIRRC skeleton for MIRIAD sun3 installation
\newline \ 
\newline \abox{File:} \$MIR/src/scripts/sun/MIRRCsun3
\newline \abox{Responsible:} Mark Stupar
\newline \abox{Keywords:} system operation
\newline{\tenpoint\newline
MIRRC skeleton for MIRIAD sun3 installation.  The script
sets MIRIAD environment variables and aliases.

Installation:  copy the MIRRCsun3 file in the scripts source
directory (sun subdirectory) file to \$MIR/MIRRC; then
manually edit that file to specify the locations of
environment variables specified below.

{\eightpoint\begintt
  Usage:  source $MIR/MIRRC

\endtt}
Environment variables to be locally specified:

{\eightpoint\begintt
  $MIR ....... location of MIRIAD's home directory
  $MIRXLIB ... location of local libX11.a (set to null
               if libX11.a is not installed)
  $MIRLXDS ... location of local X-libraries needed by
               MXDS (set to null if these are not
               installed)
  $MIRLPGP ... location of local PGPLOT library to be
               used for program loading (may be set to
               the site's separately-installed version)
\endtt}
\par}
\module{MIRRCsun3sun4}%
\noindent MIRRC skeleton for MIRIAD sun3/4 installation
\newline \ 
\newline \abox{File:} \$MIR/src/scripts/sun/MIRRCsun3sun4
\newline \abox{Responsible:} Mark Stupar
\newline \abox{Keywords:} system operation
\newline{\tenpoint\newline
MIRRC skeleton for MIRIAD installation on a systems where
MIRIAD will be installed on both sun3's and sun4's.  The
script sets MIRIAD environment variables and aliases.

Installation:  copy the MIRRCsun3sun4 file in the scripts source
directory (sun subdirectory) file to \$MIR/MIRRC; then
manually edit that file to specify the locations of
environment variables specified below.

{\eightpoint\begintt
  Usage:  source $MIR/MIRRC

\endtt}
Environment variables to be locally specified:

{\eightpoint\begintt
  $MIR ....... location of MIRIAD's home directory
  $MIRXLIB ... location of local libX11.a (set to null
               if libX11.a is not installed)
  $MIRLXDS ... location of local X-libraries needed by
               MXDS (set to null if these are not
               installed)
  $MIRLPGP ... location of local PGPLOT library to be
               used for program loading (may be set to
               the site's separately-installed version)

\endtt}
Be sure to specify the variables as appropriate for both
systems. The home MIRIAD directory \$MIR is the same for
both systems (ie, the same source code is used).
\par}
\module{MIRRCsun4}%
\noindent MIRRC skeleton for MIRIAD sun4 installation
\newline \ 
\newline \abox{File:} \$MIR/src/scripts/sun/MIRRCsun4
\newline \abox{Responsible:} Mark Stupar
\newline \abox{Keywords:} system operation
\newline{\tenpoint\newline
MIRRC skeleton for MIRIAD sun4 installation.  The script
sets MIRIAD environment variables and aliases.

Installation:  copy the MIRRCsun4 file in the scripts source
directory (sun subdirectory) file to \$MIR/MIRRC; then
manually edit that file to specify the locations of
environment variables specified below.

{\eightpoint\begintt
  Usage:  source $MIR/MIRRC

\endtt}
Environment variables to be locally specified:

{\eightpoint\begintt
  $MIR ....... location of MIRIAD's home directory
  $MIRXLIB ... location of local libX11.a (set to null
               if libX11.a is not installed)
  $MIRLXDS ... location of local X-libraries needed by
               MXDS (set to null if these are not
               installed)
  $MIRLPGP ... location of local PGPLOT library to be
               used for program loading (may be set to
               the site's separately-installed version)
\endtt}
\par}
\module{MIRRCumd}%
\noindent MIRRC csh skeleton for MIRIAD unix installation
\newline \ 
\newline \abox{File:} \$MIR/src/scripts/sun/MIRRCumd
\newline \abox{Responsible:} Peter Teuben
\newline \abox{Keywords:} system operation
\newline{\tenpoint\newline
MIRRC skeleton for MIRIAD unix installation.  The script
sets MIRIAD environment variables and aliases, assuming
the root directory of MIRIAD has been defined.

Installation:  copy the MIRRCumd file in the scripts source
directory (sun subdirectory) file to \$MIR/MIRRC; then
manually edit that file to specify the locations of
environment variables specified below.

{\eightpoint\begintt
  Usage:  source $MIR/MIRRC

\endtt}
Environment variables assumed to be present:
{\eightpoint\begintt
  $HOSTTYPE... signature of the hosttype (see also tcsh)
  $MIR ....... location of MIRIAD's home directory

\endtt}
Environment variables to be locally specified if different:
{\eightpoint\begintt
  $MIRXLIB ... location of local libX11.a (set to null
               if libX11.a is not installed)
  $MIRLXDS ... location of local X-libraries needed by
               MXDS (set to null if these are not
               installed)
  $MIRLPGP ... location of local PGPLOT library to be
               used for program loading (may be set to
               the site's separately-installed version)
\endtt}
\par}
\module{mirsubmit}%
\noindent Programmer submission of code to MIRIAD
\newline \ 
\newline \abox{File:} \$MIR/src/scripts/sun/mir.submit
\newline \abox{Responsible:} Peter Teuben
\newline \abox{Keywords:} system operation
\newline{\tenpoint\newline
Submit source code from programmers to the MIRIAD system.
Using a lockfile, source code permissions are changed after
code checkout by programmers to prevent overwriting by
another programmer.

{\eightpoint\begintt
  Usage:  mir.submit [-m] [-i] [-p] source destdir

  -m ........ move files rather than copying
  -i ........ run non-interactively
  -p ........ preserved timestamp when copying
  source .... source code files(s)
  destdir ... destination directory

\endtt}
The script requires minor system modifications and is used
only by the Maryland MIRIAD programmers.

Alias:  ``mirsubmit'' is an alias for ``mir.submit''.
\par}
\module{unMIRRC}%
\noindent Unset MIRIAD environment variables
\newline \ 
\newline \abox{File:} \$MIR/src/scripts/sun/unMIRRC
\newline \abox{Responsible:} Mark Stupar
\newline \abox{Keywords:} system operation
\newline{\tenpoint\newline
Unset MIRIAD environment variables and unalias MIRIAD
aliases (sometimes useful for internal MIRIAD support,
but not for general use).

The script performs an ``unsetenv'' on all MIRIAD
environment variables and an ``unalias'' on all MIRIAD
aliases.  Caution:  the user's search path is not
altered and any subsequent execution of the .cshrc file
will reset everything.

{\eightpoint\begintt
  Usage:  source $MIR/src/scripts/sun/unMIRRC
\endtt}
\par}
