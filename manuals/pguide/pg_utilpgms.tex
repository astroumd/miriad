%------------------------------------------------------------------------
% Utility Programs Chapter
%------------------------------------------------------------------------

A few programs have been developed to simplify the development of good
portable code.  Additional information on these programs can be found in
the specific sections of the {\bf MIRIAD Executable Modules Manual},
which documents all MIRIAD programs.

\beginsection{flint}

{\bf flint} is a FORTRAN program checker. It aims at uncovering programming
bugs and bad programming practises. Given FORTRAN source as input, {\bf flint}
produces warnings on the terminal and in a listing file ({\tt flint.log}).
{\bf flint} checks for variables which are unused, uninitialized or
undeclared. It also checks that the number, type and intent of arguments
passed to a subroutine remains consistent amongst all calls (an argument's
intent is whether its value is input to, or output from a routine).
{\ninepoint\begintt
     flint [-acdfhjrsx2?] [-I incdir] [-o file] [-l] file ...
\endtt}
There are many command line switches (though the defaults are usually
adequate).  Invoking {\bf flint} with the {\tt ?} switch causes it to
print out brief information about {\bf flint}, and all the command line
switches. For example:
{\ninepoint\begintt
     flint -?
\endtt}
will give some help information.

\beginsub{Making flint Quieter}

{\bf flint} assumes that the programmer follows a fairly strict programming
convention. If the programmer does not, then {\bf flint} can produce a
voluminous number of warning messages. There are a number of switches which
help quiet {\bf flint} to manageable proportions:

\item{-c} Allow interwoven continuation and comment lines. Normally,
{\bf flint} generates warnings about this (even though it is standard
FORTRAN).  An unpleasant side effect of allowing interwoven comment and
continuation lines is that comments are not copied to the listing file.

\item{-d} Do not warn when variables are not explicitly declared. Normally
{\bf flint} insists that all variables must be explicitly declared with a
`REAL', `INTEGER', etc, statement.

\item{-f} Do not generate warnings about lines longer than 72 characters,
or lines with an odd number of single quotes. When checking if a line is
longer than 72 characters, {\bf flint} interprets tabs as if the equivalent
number of spaces had been typed (the VMS compiler treats a tab as a single
character, rather than as multiple spaces).

\item{-h} By default, {\bf flint} does not understand hollerith data. The
{\tt -h} flag instructs {\bf flint} to recognize hollerith data as integers.

\item{-j} Do not check if a variable has been initialized before being used.
If a variable is not in a DATA, COMMON, PARAMETER or SAVE statement,
and if it is not a dummy subroutine argument, then {\bf flint} normally
attempts to check that the variable is initialized before it is used. It
does this by checking that it is assigned to at a earlier line in the
routine than where it is first used. This is not necessarily correct in
routines with EQUIVALENCE statements, a maze of gotos, or which have
conditional blocks within loops.  Applying this same rule to subroutine
arguments provides {\bf flint} with a technique for determining subroutine
argument intent.  Disabling initialization checking will also disable these
intent-determining rules, and so should eliminate spurious warnings about
inconsistent subroutine argument intent.

\item{-r} Do not warn about seemingly redundant variables. Redundant variables
are those that are assigned to but never otherwise used. A common
source of most messages is when a value returned by a subroutine
is never used.  For example, {\bf flint} will complain if you ignore values
returned in an array used by a subroutine for scratch storage, or if values
returned by a general subroutine are ignored in specific cases.

\item{-s} Load definitions of FORTRAN-IV and specific intrinsic functions.
By default, {\bf flint} only knows about the standard FORTRAN-77 generic
intrinsic functions (e.g. COS, REAL, MAX). It does not know specific
function names or obsolete FORTRAN-IV functions (e.g. DCOS, FLOAT, AMAX0).
The `s' flag causes {\bf flint} to load the definitions of these as well.

\item{-u} Do not generate a warning about variables which do not seem to be
used. By default, {\bf flint} produces such a warning for local variables
(but not COMMON or PARAMETER variables).

\item{-x} Allow extended subroutine and variable names. By default {\bf flint}
warns if subroutine or variable names are longer than 8 characters, or if
they contain underline or dollar characters. Using this flag allows names
of arbitrary length, with dollar and underline characters.

\beginsub{Other Flags and Arguments}

\item{a} This causes flint to include a crude cross-reference map in its
log file.
\item{I} This flag is used to give the name of a directory to search for
include files. When it encounters an `include' statement, flint first checks
the local directory for the include file, and then checks other
directories given by the {\tt -I} switch, in the order they were present
on the command line.
\item{?} This causes some help information to be printed.

\beginsub{Bugs and Shortcomings}

{\bf flint} has a number of bugs and
shortcomings, in addition to those described above. {\bf flint} is intended
to be run on a FORTRAN that a compiler would accept.  During the early
stages of program development, a good FORTRAN compiler is a far better
tool to find programming problems. {\bf flint} bugs and shortcomings include:

\item{$\bullet$} EQUIVALENCE and IMPLICIT statements are ignored.

\item{$\bullet$} A number of archaic, poor or non-standard FORTRAN features
are not recognized. These include the ASSIGN statement and assigned GOTO,
alternate return statements, PAUSE, ENTRY, DECODE, ENCODE, many
archaic forms of i/o statements, NAMELIST and many VMS extensions.

\item{$\bullet$} While BYTE, INTEGER*2, INTEGER*4, REAL*4, REAL*8, etc.,
 statements are recognized, a warning message is generated, {\bf flint}
otherwise treats these variables as either standard INTEGERs or REALs.

\item{$\bullet$} Columns 72 to 80 of an input line (if they are present)
are assumed to be part of the FORTRAN statement, rather than a comment
field. {\bf flint} generates a warning if a line contains more than 72
characters.

{\bf flint} recognizes the following extensions, and digests them without
complaint:
\smallskip
\item{$\bullet$} DO/ENDDO and DOWHILE/ENDDO constructs.

\item{$\bullet$} INCLUDE statements.

\item{$\bullet$} Lower case characters are considered equivalent to upper
case. Tabs are treated as if they were the equivalent number of spaces.
The full printable ascii character set can appear in character strings.

\item{$\bullet$} A comment line starting with the hash character (\#).

\item{$\bullet$} The INTENT statement (see the section on Interface
Definition Libraries).

\beginsub{Intent of Subroutine Arguments}

Whenever {\bf flint} finds a call to a subroutine, or the source of a
subroutine, it attempts to determine whether a subroutine argument is
either input or output (this is called an arguments intent). Usually,
{\bf flint} will build up its knowledge of a particular routines
arguments by analyzing many uses of the subroutine, and generate a
warning whenever inconsistent use is noted.

Note that {\bf flint} expects an argument's intent to be either input or
output or input/output. Some routines use an argument as input in some
situations and output in others. Such routines may confuse {\bf flint},
and it may well generate spurious messages about the variable in the
subroutine call being uninitialized or redundant, or that the argument's
intent is inconsistent.

Normally {\bf flint} performs a single pass, analyzing files in the order
that they are given on the command line. When analyzing the early part of
the input source, {\bf flint} knowledge of argument intent is very poor.
Thus some problems (uninitialized and redundant variables) may be missed.
There are two recipes to partially avoid this. Firstly files on the
command line should be ordered with the files containing low level
subroutines first, and high level programs and application code last
(generally interface definition libraries should go first). Additionally
inside files, the lower level routines should be first, and higher level
routines last. Secondly, the user can ask {\bf flint} to perform two
passes of the input files, by using the {\tt -2} flag.  Two passes will
not necessarily correctly determine the intent of all arguments (in
general, {\tt N} passes will always be enough for programs with subroutine
calls to a depth of {\tt N}).

The following contrived example shows the worst case sort of behavior,
where {\bf flint} builds up its knowledge of argument intent quite slowly.
{\ninepoint\begintt
     subroutine a(x)
     real x
     call b(x)
     end
     subroutine b(x)
     real x
     call c(x)
     end
     subroutine c(x)
     real x
     write(*,*)x
     end
\endtt}
It would require three passes for {\bf flint} to determine that the
intent of argument {\tt x} in subroutine {\tt a} was input (the first
pass would determine the intent of {\tt x} in {\tt c}, the second would
determine the intent of {\tt x} in {\tt b}). Usually {\bf flint} determines
argument intents by other means than just allowing information to bubble
up from the lowest level to the higher level routines. But in the above
example, none of these could be applied. One of the techniques is
derived from initialization checking (described under the {\tt -i} flag).
When determining intent accurately is important, and if the initialization
checking algorithm is failing (i.e. producing spurious messages), then
initialization checking should be disabled.

Small changes can significantly change the number of passes needed to
uncover all intent information. For instance, in the example above,
if the ordering of the routines was reversed (i.e. {\tt c} first and {\tt a}
last), or if a piece of code such as
{\ninepoint\begintt
      call a(1.0)
\endtt}
appeared before subroutine {\tt a}, then {\bf flint} would require only
one pass.

\beginsub{Interface Definitions Libraries}

It is a common (normal) situation
in program development to have a library of standard functions and
routines. Though {\bf flint} knows the interface definitions (i.e.,
number, type and intent of each argument) of standard FORTRAN
functions, it is desirable for users to build and use a file (or
files) describing the interfaces to their own library.  This section
describes the features in {\bf flint} to achieve this.

{\bf flint} can output a text file, in pseudo-FORTRAN, which contains
interface descriptions of all the routines that it has encountered in
this run. The format is:
{\ninepoint\begintt
     flint -o file ....
\endtt}
Each routine in the output file will probably contain many `{\tt INTENT}'
statements (part of the FORTRAN-8X standard). INTENT is used to indicate
whether a subroutine argument is input to or output from a routine (or both),
or whether this is unknown. Typically a routine description would look
like:
{\ninepoint\begintt
      subroutine example(arg1,arg2,arg3,arg4)
      real arg1,arg2
      integer arg3,arg4
      intent(in) arg1,arg3
      intent(out) arg3,arg4
      intent(unknown) arg2
      end
\endtt}
Here, {\bf flint} believes {\tt arg1} is input, {\tt arg4} is output,
and {\tt arg3} is both input and output.  {\bf flint} could not
determine whether {\tt arg2} was input or output.

The user can edit an interface file created by {\bf flint}, or create an
entire interface description by hand.

Such an interface description file appears as normal FORTRAN to {\bf flint},
so it can be input to {\bf flint} (to teach {\bf flint} about a user's
standard routine interfaces) on subsequent uses of {\bf flint}. While
these interface definitions could be input along with normal source files,
preceding the interface file with the {\tt -l} (library) switch has the
advantage that the source of this file does not appear in the listing
file. Warning messages generated by analyzing this file are also suppressed.

The following example generates an interface description file,
{\tt library.dat}, from source files {\tt routine1.for}, {\tt routine2.for},
and {\tt routine3.for}.  The interface definition is then used in a
subsequent {\bf flint} run on {\tt program.for}.
{\ninepoint\begintt
     flint -o library.dat routine1.for routine2.for routine3.for
     flint -l library.dat program.for
\endtt}

\beginsection ratty

{\bf ratty} is a FORTRAN preprocessor, intended to simplify the
job of developing code that is to run on several machines.  {\bf ratty}
checks for some non-ANSI or dubious programming practises; secondly, it
converts some FORTRAN extensions into standard FORTRAN; and thirdly, it
handles some compiler directives associated with conditional compilation
or code optimization.

Apart from the extra trouble of running the preprocessor before
compilation,  preprocessors are generally a nuisance, in that they make
the code that is developed and the code that is debugged different
(both compile and run time debugging).  This tends to complicate the
debugging process. To this end, {\bf ratty} attempts to make the minimum
changes necessary to code to get it to compile.  Comments and indentation
are retained in the output, so that the output should be nearly as
readable as the input.  Additionally, when conditional compilation
directives are avoided, it will be found that much code does not need
to be preprocessed on a number of compilers.

\beginsub{The Command}

The command is:
{\ninepoint\begintt
     % ratty [-s system] [-I incdir] [-D symbol] [-b] [-?] input output
\endtt}
``Input'' is a text file containing a mildly extended FORTRAN, whereas
``output'' is the resultant standard (?) code. The default input and
output are the standard input and standard output. A few command line
flags are also recognized:

\item{-s} The string following this flag indicates the target compiler.
{\bf ratty} performs some special processing for the following
compilers: \newline
{\parskip=0pt
\def\bbox#1{\hbox to 2.0cm{#1\hfil}}
\bbox{\tt vms} VAX/VMS FORTRAN compiler. \newline
\bbox{\tt cft} Cray FORTRAN compiler (cft77). \newline
\bbox{\tt f77} Generic UNIX FORTRAN-77 compiler. \newline
\bbox{\tt convex} Convex C-1 compiler. \newline
\bbox{\tt fx} Alliants compiler. \newline
\bbox{\tt trace} Multiflow Trace computers. \newline
}

{\bf ratty} assumes that target compilers other than these are strict
FORTRAN-77 compilers.

\item{-I} The string following this flag indicates an alternate directory to
search for include files. The -I flag can occur several times, giving several
directories. When opening include files, first the current directory is
checked, and then each directory given by the -I flag is check in the order
in which they appeared on the command line.

\item{-D} The name following this flag is treated as if it appeared in a
\#define statement. Note that, unlike the cc compiler, a space is required
between the -D and the name.

\item{-b} If this flag is given, every backslash in the input is converted to
two backslashes in the output. This is useful when the target compiler
treats backslash as an ``escape character'' (i.e. several UNIX FORTRAN
compilers).

\item{?} This causes some help (on using {\bf ratty}) to be printed.

\beginsub{Language Extensions}

{\bf ratty} converts the following language extensions into
standard FORTRAN, where necessary.

\item{$\bullet$} Tab characters are replaced with the corresponding
number of blanks.

\item{$\bullet$} Both DO/ENDDO and DOWHILE/ENDDO are replaced with the
standard FORTRAN-77 loops.

\item{$\bullet$} The IMPLICIT NONE and IMPLICIT UNDEFINED statements
may be converted to the form required by the local compiler, or
eliminated altogether.

\item{$\bullet$} INCLUDE files are expanded.

\beginsub{Optimization Directive}

Vector computers occasionally need
directives to help optimize some loops. The needed directives, however,
vary between manufacturers. {\bf ratty} takes optimization directives
in a standard form and converts them to the form recognized by the
target compiler. A directive is introduced by either a {\tt c\#} or a
{\tt \#} starting a line (i.e. starting in column 1), for example:
{\ninepoint\begintt
  c#directive
\endtt}
or
{\ninepoint\begintt
  #directive
\endtt}
Either form can be used, but the {\tt c\#} form is preferable, as the
directive will be seen as a comment to a standard compiler. The
directives are:

\item{$\bullet$} {\bf ivdep} instructs the compiler to ignore any apparent
vector dependencies in the immediately following loop.  This is a commonly
used directive.

\item{$\bullet$} {\bf maxloop nn} indicates that the loop count of
the immediately following loop will not exceed `nn'. This enables the
compiler to optimize some short loops. For example
{\ninepoint\begintt
  c#maxloop 32
      do i=1,n
         .
         .
      enddo
\endtt}
indicates that the do loop will not execute more than 32 times.

\item{$\bullet$} The {nooptimize} directive appears just before a
do-loop that should not be optimized.

\beginsub{Conditional Compilation Directives}

When coding a task for
several machines, it is sometimes desirable that different code is
executed for different machines. For example, to achieve good
efficiency, the inner loops may need to be different on a scalar
and vector machine.

The basic syntax is the same as the optimization directives (i.e.
{\tt c\#} or {\tt \#} starting a line, followed by the directive). However,
as code using conditional compilation will have to be passed through
{\bf ratty} to produce the correct code, the {\tt \#} form is preferred.
This will be seen as an error by a standard compiler, thus preventing
code from accidently not being passed through the preprocessor.

There are four directives of interest, namely {\tt ifdef, ifndef,
else} and {\tt endif}. Like the cpp preprocessor, {\tt ifdef} and
{\tt ifndef} pass the following section of code to the compiler if
a particular ``symbol'' is defined or not.  Currently there are only
one or two symbols that are defined. Firstly, a symbol with the
compiler target name is defined (the target compiler name is gained
from the command line {\bf -s} flag). Secondly, a symbol `vector' is
defined if the target computer is a vector machine. For example,
 the command line:
{\ninepoint\begintt
   % ratty -s cft input output
\endtt}
causes the symbols `cft' and `vector' to become defined.  Then, given
the following code fragment from ``input'', the first section is
copied to ``output'', whereas the second is discarded.
{\ninepoint\begintt
     #ifdef cft
        .
        .  Use this code on the Cray.
        .
     #else
        .
        .  This code on all other machines.
        .
     #endif
\endtt}
