All MIRIAD datasets are, in reality, directories containing
files.  These files contain the actual data.  A reference to a
MIRIAD dataset always refers to the directory that contains
the files, and not to the files themselves.  One of the files
contained in a MIRIAD dataset MUST be named ``header''
(when there is no ``header'' information, the file still exists,
but contains 0 bytes of data).

All MIRIAD datasets use the same format to store data, whatever
the representation of particular storage types on any given machine.
Thus, for example, ``real'' data is stored as 32 bit IEEE on all
machines even though a ``real'' might be 64 bit IEEE on some machines
(for example, the Cray).  MIRIAD routines exist to pack and unpack
the data as appropriate to the machine on which MIRIAD will be
run, while any actual computations are done to the precision of the
particular machine in question.

This means that any MIRIAD dataset is completely portable to
any other machine on which MIRIAD is supported.  Possible MIRIAD
storage types for all data consist of the following:

{\ninepoint\begintt
char ...... null-terminated character string
string .... character string, not null-terminated
real ...... (32 bit IEEE)
double .... (64 bit IEEE)
complex ... (2 * 32 bit IEEE)
integer ... (32 bit 2's complement)
short ..... (16 bit 2's complement)
\endtt}

MIRIAD supports both {\bf visibility} datasets and {\bf image}
datasets, and though each MIRIAD dataset may not contain every
possible file, each must contain a ``header''.  Their formats are
discussed below.  An effort is made to discuss only the files, rather
than the MIRIAD subroutines that actually use them; when
necessary, a reference to a sourcecode file containing MIRIAD
subroutines will be given in all-caps, without the filename extension
(example:  a reference to a routine in \$MIR/src/subs/uvio.c will be
made as {\tt UVIO}).

\beginsection MIRIAD Data Items

In general, MIRIAD programmers need not worry about which
particular file contains which particular data item:  the lowest
level data item is referenced by name, and MIRIAD's subroutines
find the information, largely relieving programmers from memorizing
where things are specifically located.

Dataset item names have the following characteristics:
\item{$\bullet$} no more than 8 lower-case characters, not counting
terminators such as null characters.
\item{$\bullet$} contain no underscore or dollar-sign characters.
\item{$\bullet$} follow conventions in determining names (for example,
item name ``image'' is always used to store the pixel data of an image;
``naxis'' is the number of dimensions in an image).
\item{$\bullet$} new data items can be invented as the need arises.
\item{$\bullet$} every item is not stored in every dataset; only those
data items that are pertinent to a given dataset are stored in that
dataset.

Listings of existing uv Variables and Image Items can be found in the
appendices.

\beginsection MIRIAD uv Datasets

{\raggedright
\def\bbox#1{$\underline{\smash{\hbox{#1}}}$}
\ninepoint
\tabskip=0em
\halign {#\tabskip=2em&#\tabskip=2em&#\hfil\cr

\bbox{File name}&\bbox{Type}&\bbox{Description}\cr
\cr
header      & binary  & ``short'' data items\cr
history     & text    & history text file (in principle editable) \cr
vartable    & text    & lookup table for all uv variables (do not edit!) \cr
visdata     & binary  & mixed-mode data stream of uv variables \cr
flags       & integer & flags for narrow band data (optional) \cr
wflags      & integer & flags for wideband data (optional) \cr
}
}

\beginsub{header}

The {\tt header} file contains ``short'' data items (such as the number
of antennas).  If no {\tt header} data exists, then the file still exists,
with 0 bytes of data.

The format is undocumented.  Examples seen to date have multiples of 32
bytes, where the last 8 bytes of the last {\tt header} item are discarded
when they contain no information (that is, the last {\tt header} item
may consist of 24 bytes.  For those particulars not described, it is not
unreasonale to base guesses on alignment requirements.  Formats observed
to date are consistent with the following for each {\tt header} item:

{\raggedright
\vskip .2in
\def\bbox#1{$\underline{\smash{\hbox{#1}}}$}
\ninepoint
\tabskip=0em
\halign {#\tabskip=2em&#\tabskip=2em&#\hfil\cr

\bbox{Bytes}&\bbox{Type}&\bbox{Description}\cr
\cr
00-09   & char    & uv variable name\cr
10-11   & -       & not used\cr
12-15   & integer & code\cr
        &         & = 6 + strlen for char data\cr
        &         & = 10 for real and integer\cr
        &         & = 20 for double\cr
        &         & = ?  for others\cr
16-19   & integer & code for data type\cr
        &         & = 1 for char\cr
        &         & = 2 for integer\cr
        &         & = 3 for short\cr
        &         & = 4 for real\cr
        &         & = 5 for double\cr
        &         & = 6 for complex\cr
        &         & = 7 for string\cr
20-31   & (data)  & char, int, and real begin\cr
        &         & in byte 20; double begins\cr
        &         & in byte 24.  Character data\cr
        &         & is not null-terminated, its\cr
        &         & stringlength determined in\cr
        &         & bytes 12-15.\cr
}
}

\beginsub{vartable - The Variable Table}

This model of variables is implemented using two items, {\tt vartable} (a text
file) and {\tt visdata} (a binary file). {\tt Vartable} gives an ordered list
of the names and types of the variables present in a particular uv dataset.
A simple {\tt vartable} file might look like:

{\ninepoint\begintt
j corr
d coord
t time
i nchan
r baseline
\endtt}

Here, the first character gives the data type in which the variable is
stored on disk. This will be one of `a', `j', `i', `r', `d' or `c',
indicating ascii, 16-bit integer, 32-bit integer, 32-bit floating point,
64-bit floating point, or 64-bit complex data respectively.  Then follows
the name of the variable. The list is ``ordered'' in the sense that the
first line of {\tt vartable} describes variable number 0, the second line
describes variable number 1, etc. The {\tt vartable} item is the only
place where a variable's name and type are given -- elsewhere the variable
number is used. Note that {\tt vartable} does not give the number of elements
in a variable -- this is allowed to change within a data-set, and so is
encoded elsewhere.

For an old data-set, reading {\tt vartable} is the first operation performed
by the {\tt uvopen} routine. For a new data-set, writing {\tt vartable}
is one of the last operations before completing the close of the data-set.

{\raggedright
\vskip .2in
\def\bbox#1{$\underline{\smash{\hbox{#1}}}$}
\ninepoint
\tabskip=0em
\halign {#\tabskip=2em&#\tabskip=2em&#\hfil\cr

\bbox{Bytes}&\bbox{Type}&\bbox{Description}\cr
\cr
00-00   & char    & `a' for char data\cr
        &         & `j' for short data\cr,
        &         & `i' for integer data\cr
        &         & `r' for real data\cr
        &         & `d' for double data\cr
        &         & `c' for complex data\cr
01-01   & char    & a blank character\cr
02-     & char    & the name of the variable,\cr
        &         & all but the last line terminated\cr
        &         & by a ``newline'' character.\cr
}
}

\beginsub{visdata - The Actual Data}

The {\tt visdata} item contains the actual values of the variables. It is
an item of mixed data types. A {\tt visdata} item can be viewed as consisting
of records and subrecords. A record contains all the changes to variables,
etc, which change simultaneously. It will typically consist of many subrecords.

A subrecord consists of a four-byte subrecord header,
possibly followed by a value. The four bytes consist of a byte giving the
subrecord type, a byte giving the number of the variable affected (if
any), and two bytes which are spare. These spare bytes must be zero. They
are there to help keep subrecord headers on four byte boundaries. They may
also be used in the future if more than 256 variables are needed (i.e.,
more than 1 byte is needed to give the variable number). There are three
types of subrecords:
\item{$\bullet$} A subrecord giving the size of a variable. The value of
this subrecord is a 32-bit integer, giving the variable's size in bytes,
as it exists on disk.
\item{$\bullet$} A subrecord giving the value of a variable. The size of the
value field of this subrecord is given by a preceding ``size'' subrecord.
The value may also contain extra padding to ensure that {\tt HIO}'s
alignment requirements are met, and that a subrecord starts on a four byte
boundary. The size given in a ``size'' subrecord does not include such padding.
\item{$\bullet$} An end-of-record subrecord. This is a marker indicating the
end of this record. When reading from {\tt visdata} (performed by
{\tt uv\_scan}), the {\tt UVIO} routines always read to the end of a record
before they return.
Because of the size of subrecords, and the number of subrecords in a
record, are not fixed, the {\tt visdata} item must be processed in a sequential
manner.


\beginsub{history}

Most datasets will have an associated history item.  This is a text file
containing a description of the dataset and the processing that has been
performed on it.  When a MIRIAD dataset which already has a history
file is processed, it is customary to append the new information to the
old history, thus providing a complete history of the dataset.

For uniformities sake, history comments follow a standard format. The
following is an example of the history comments written by the task
DEMOS:

{\ninepoint\begintt
DEMOS: Miriad DeMos (version 30-apr-90 rjs)'
DEMOS: Input parameters are
DEMOS:   vis=uvn
DEMOS:   map=mosclean
DEMOS: Pointing offset used (arcsec):   24.0 -30.0
\endtt}

Note that all comments start with the task name. The first comment gives a
version date of the program. The next three lines were generated by
subroutine {\tt hisinput}, and the last line was an extra comment.

The following example shows the history of a dataset's processing:

{\ninepoint\begintt
UVHAT: Translated by UVHAT (version 1.0 09-Nov-90)
UVHAT: File            ant         weight          delphi            sca
newhat: CHICYGR.05MAR  123   1.00  1.00  1.00   0.0  0.0  0.0   1.00  1.
UVCAT: Miriad UvCat (version 1.0 25-Jan-91)
UVCAT: Command line inputs follow:
UVCAT:   vis=new
UVCAT:   select=window(1)
UVCAT:   out=junk
AIPS WTSCAL =  3.01425438374E-02
FITS (MIRIAD)
FITS: Input FITS file = uvfits
FITS: Output MIRIAD file = fitsuv
FITS: Stokes parameter = 0
\endtt}

\beginsub{flags}

Discussion: Flags for narrow band data, if they are present.  The flags
specify whether data is described as ``good'' data or ``bad'' data.
The absence of this file implies that all data is ``good'' data.

File format:  not available.

\beginsub{wflags}

Discussion: Flags for wideband data.  The flags specify whether data is
described as ``good'' data or ``bad'' data.  The absence of this file
implies that all data is ``good'' data.

File format:  not available.


\beginsection MIRIAD Image Datasets

{\raggedright
\def\bbox#1{$\underline{\smash{\hbox{#1}}}$}
\ninepoint
\tabskip=0em
\halign {#\tabskip=2em&#\tabskip=2em&#\hfil\cr

\bbox{Item name}&\bbox{Type}&\bbox{Description}\cr
\cr
header      & binary  & ``short'' data items\cr
history     & text    & history text file (in principle editable) \cr
image       & binary  & multi-dimensional pixel data\cr
flags       & integer & flags for the image data\cr
}
}

\beginsub{header}

Image dataset header files follow the same format as uv dataset header
files.

\beginsub{history}

Image dataset history files follow the same format as uv dataset history
files.

\beginsub{image}

This is the multi-dimensional image.  The first 4 bytes of the file are
an integer code specifying the type of data represented.  The remainder
of the file is the data.

{\raggedright
\vskip .2in
\def\bbox#1{$\underline{\smash{\hbox{#1}}}$}
\ninepoint
\tabskip=0em
\halign {#\tabskip=2em&#\hfil\cr

\bbox{Code}&\bbox{Type of Data}\cr
\cr
1 & character data (one byte per data point)\cr
2 & integer data (four bytes per data point)\cr
3 & short data (two bytes per data point)\cr
4 & real data (four bytes per data point)\cr
5 & double data (eight bytes per data point)\cr
6 & complex data (eight bytes per data point)\cr
7 & string data (one byte per data point)\cr
}
}

\beginsub{flags}

Discussion: Flags for the pixel data.  The flags specify whether data is
described as ``good'' data or ``bad'' data.  The absence of this file
implies that all data is ``good'' data.

File format:  not available, but the file is logical, with every data
point corresponding to a particular point within the pixel data, and
with the value of the bit setting determining whether a pixel data point
has been flagged ``good'' or ``bad''.
