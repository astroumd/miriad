MIRIAD subroutines are generally only callable by FORTRAN code,
and this should be the language of choice for MIRIAD programmers.
There are, however, C-callable entry points to most I/O routines. In certain
applications (particularly ``special'' task HCCONV) these may be of some
use (their use is, however, discouraged).

The calling sequence is similar to the FORTRAN calling sequence.
There are some differences:

{\bf entry point name}:  The C entry point name is formed by adding
{\bf \_c} to the FORTRAN name. The entry point names are always lower case.
For example, the FORTRAN routine {\bf bug} becomes {\bf bug\_c}.

{\bf data type}:  Most scalar arguments that are only passed into a
subroutine (but not returned) are passed by value, rather than by address.
Implementations of FORTRAN normally pass all arguments by address.

{\bf char strings}:  FORTRAN character variables are usually passed as the
address of a null-terminated array of type ``char''. Occasionally, when
the subroutine expects only a single character, it is passed by value.

{\bf complex data}:  Complex data is handled as an array of two ``float''
arrays (real part followed by the imaginary part).

{\bf boolean data}:  Logical data is treated as integers (0 representing
false, and non-zero representing true).  WARNING:  ``true'', ``false'',
and ``NULL'' have been redefined in some subroutines.  So far as has
been noticed, they have been redefined to their expected values.

{\bf returned char strings}:  When the routine passes back a character
string, there will be an extra argument ({\bf length} or {\bf n} in the
descriptions below) which give the size of the return buffer.

Below is a summary of the C-callable entry points, in pseudo-C.  Take
note:  the listing may have some small errors, perhaps typographical, so
be aware when using them.

{\ninepoint\begintt
void bug_c(char sev,char message[])
void bugno_c(char sev,int status)
\endtt}
{\ninepoint\begintt
void haccess_c(int *tno,void **item,char name[],char status[],int *iostat)
void hclose_c(int tno)
\endtt}
{\ninepoint\begintt
void hdaccess_c(void *item,int *iostat)
void hdcopy_c(int tin, int tout, char itemname[])
void hdprobe_c(int tno,char keyword[],char descr[length],int length,
     char type[],int *n)
int hdprsnt_c_c(int tno, char itemname[])
\endtt}
{\ninepoint\begintt
void hisclose_c(int tno)
void hisopen_c(int tno,char status[])
void hisread_c(int tno,char line[length],int length,int *eof)
void hiswrite_c(int tno,char line[])
\endtt}
{\ninepoint\begintt
void hopen_c(int *tno,char name[],char status[],int *iostat)
  (Note: The hread routines are implemented as macros)
void hreada_c(void *item,char line[length],int length,int *iostat)
  (Note: In hreadb, buffer is not zero-terminated)
void hreadb_c(void *item,char *buffer[],int offset,int length,int *iostat)
void hreadd_c(void *item,double buffer[],int offset,int length,int *iostat)
void hreadi_c(void *item,int buffer[],int offset,int length,int *iostat)
void hreadj_c(void *item,int buffer[],int offset,int length,int *iostat)
void hreadr_c(void *item,float buffer[],int offset,int length,int *iostat)
int hsize_c(void *item)
\endtt}
{\ninepoint\begintt
  (Note: The hwrite routines are implemented as macros)
void hwritea_c(void *item,char line[length],int length,int *iostat)
  (Note: In hwriteb, buffer need not be zero-terminated)
void hwriteb_c(void *item,char *buffer[length],int offset,int length,
     int *iostat)
void hwrited_c(void *item,double buffer[],int offset,int length,int *iostat)
void hwritei_c(void *item,int buffer[],int offset,int length,int *iostat)
void hwritej_c(void *item,int buffer[],int offset,int length,int *iostat)
void hwriter_c(void *item,float buffer[],int offset,int length,int *iostat)
\endtt}
{\ninepoint\begintt
  (Note: There is no rdhdr_c or rdhdi_c. Implement them with rdhdd_c)
void rdhda_c(int tno,char *keyword[],char value[length],char defval[],
     int length)
void rdhdc_c(int tno,char keyword[],float *value,float defval[2])
void rdhdd_c(int tno,char keyword[],double *value,double defval)
\endtt}
{\ninepoint\begintt
void scrclose_c(void *handle)
void scropen_c(void **handle)
void scrread_c(void *handle,float buffer[],int offset,int length)
void scrwrite_c(void *handle,float buffer[length],int offset,int length)
\endtt}
{\ninepoint\begintt
void uvclose_c(int tno)
void uvcopyvr_c(int tin,int tout)
void uvflgwr_c(int tno,int flags[])
  (Note: The uvgetvr routines are implemented as macros)
void uvgetvra_c(int tno,char var[],char data[n],int n)
void uvgetvrc_c(int tno,char var[],float data[2*n],int n)
void uvgetvrd_c(int tno,char var[],double data[n],int n)
void uvgetvri_c(int tno,char var[],int data[n],int n)
void uvgetvrj_c(int tno,char var[],int data[n],int n)
void uvgetvrr_c(int tno,char var[],float data[n],int n)
\endtt}
{\ninepoint\begintt
void uvinfo_c(int tno,char object[],double data[])
void uvmark_c(int tno, int flag)
void uvnext_c(int tno)
void uvopen_c(int tno,char name[],char status[])
\endtt}
{\ninepoint\begintt
  (Note: The uvputvr routines are implemented as macros)
void uvprobvr_c(int tno,char var[],char *type,int *length,int *updated)
  (Note: In uvputvra string does not need to be zero terminated).
void uvputvra_c(int tno,char var[],char string[n],int n)
void uvputvrc_c(int tno,char var[],float data[2*n],int n)
void uvputvrd_c(int tno,char var[],double data[n],int n)
void uvputvri_c(int tno,char var[],int data[n],int n)
void uvputvrr_c(int tno,char var[],float data[n],int n)
\endtt}
{\ninepoint\begintt
  (Note: The uvrdvr routines are implemented as macros)
void uvrdvra_c(int tno,char var[],char data[length],char def[],int length)
void uvrdvrc_c(int tno,char var[],float data[2],float def[2])
void uvrdvrd_c(int tno,char var[],double *data,double def)
void uvrdvri_c(int tno,char var[],int *data,int def)
void uvrdvrr_c(int tno,char var[],float *data,float def)
void uvread_c(int tno,double preamble[4],float data[2*n],int flags[n],
     int n,int *nread)
\endtt}
{\ninepoint\begintt
void uvrewind_c(int tno)
int uvscan_c(int tno,char var[])
void uvselect_c(int tno,char object[],double p1,double p2,int datasel)
void uvset_c(int tno,char object[],char type[],int n,float p1,float p2,
     float p3)
void uvtrack_c(int tno,char name[],char switches[])
\endtt}
{\ninepoint\begintt
int uvupdate_c(int tno)
void uvwflgwr_c(int tno,int flags[])
void uvwread_c(int tno,float data[2*n],int flags[n],int n, int *nread)
void uvwrite_c(int tno,double preamble[4],float data[2*n],int flags[n],
     int n)
void uvwwrite_c(int tno, float data[2*n],int flags[n],int n)
\endtt}
{\ninepoint\begintt
void wrhda_c(int tno,char keyword[],char value[])
void wrhdc_c(int tno,char keyword[],float value[2])
void wrhdd_c(int tno,char keyword[],double value)
void wrhdi_c(int tno,char keyword[],int value)
void wrhdr_c(int tno,char keyword[],float value)
\endtt}
{\ninepoint\begintt
void xyclose_c(int tno)
void xyflgrd_c(int tno,int index,int buffer[])
void xyflgwr_c(int tno,int index,int buffer[])
void xymkrd_c(int tno,int index,int buffer[],int n,int *nread)
void xymkwr_c(int tno,int index,int buffer[],int n)
void xyopen_c(int *tno,char name[],char status[],int naxis,int axes[])
void xyread_c(int tno,int index,float buffer[])
void xysetpl_c(int tno,int naxis,int axes[])
void xywrite_c(int tno,int index,float buffer[])
\endtt}
