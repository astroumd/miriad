%------------------------------------------------------------------------
% Chapter 11 - Routine system maintenance
%------------------------------------------------------------------------

This chapter explains how to port MIRIAD from an existing site to another
site, how to set up the new site, and how to maintain MIRIAD at the site.

MIRIAD is currently installed and supported on {\tt Sun3}, {\tt Sun4}, and
{\tt Convex} systems.  In the past, it has been (and may still be, albeit
not formally supported) installed on {\tt Cray}, {\tt Alliant}, {\tt VAX},
and {\tt Multiflow} systems.  The machines chosen reflect current patterns
of usage, rather than a particular orientation or disaffection for any
particular machine.  MIRIAD is set up to be (and to look like) MIRIAD, no
matter which machine is under discussion:  a user familiar with MIRIAD on
one system can port MIRIAD datasets as-is to another platform and 
immediately use those datasets on the new platform's MIRIAD system.
Moreover, MIRIAD will look the same (identically the same) to the user on
both systems:  the same interface, the same programs, the same everything.
MIRIAD may be ported from any machine to any other machine ({\it e.g.},
port MIRIAD from a {\tt Sun} to a {\tt Convex}) without incident.

MIRIAD's likely future course will be to aim towards support on {\tt UNIX}
platforms having {\tt IEEE 754} data format.  Existing support for other
platforms ({\it e.g.}, for {\tt VMS}' data format) is intended to remain,
though it will likely become out of date at some point of time in the
future.

\beginsection{Setting up a MIRIAD Site}

This section discusses the porting of MIRIAD from an existing site, and
setting MIRIAD up at a new site.

\beginsub{Porting MIRIAD from an Existing Site}

Make a tarfile of the essential parts of the system.  You'll need about
16 MB for the tarfile (you can compress it down to about 8 MB, if
necessary).
{\ninepoint\begintt
    cd $MIR
    tar -cf /home/mydir/miriad.tar borrow cat manuals src
\endtt}
Port this tarfile to the new site.
{\ninepoint\begintt
    cd /home/mydir
    ftp new.site
        [enter logonid]
        [enter password]
        [enter "bin" for binary mode transmission]
        [enter "put miriad.tar"]
        [enter "quit"]
\endtt}

\beginsub{Setting up MIRIAD at a Site}

Preconditions:
\item{$\bullet$} You must have the ``usual'' system binary directories in
your searchpath (to use such things as awk).  All MIRIAD users must also
have these directories in their directory searchpaths, as well.
\item{$\bullet$} You must not have MIRIAD environment variables set.
\item{$\bullet$} This procedure assumes the use of {\tt csh}, and that user
directory searchpaths are set in the {\tt .cshrc} file, not in the
{\tt .login} file.
\item{$\bullet$} It is important to use fully qualified directory names,
not relative ones; and to list them relative to the automounter, if this
is how the system is set up.
\item{$\bullet$} For purposes of this example:  assume that the logon of
the MIRIAD owner is {\tt miriad}; and that that logon's home directory
is {\tt /home/babbage/miriad}; and that installation is to be done on a
{\tt Sun4}.

Create the directory that will be MIRIAD's home directory.  Untar the
transmitted tarfile and create MIRIAD's directory structure.  Environment
variable {\tt \$MIRTEMP} is used just to help you set things up.  The
{\tt mir.mkdirs} script creates all the necessary directories within
the MIRIAD tree.
{\ninepoint\begintt
    setenv MIRTEMP /home/babbage/miriad/miriad.home
    mkdir $MIRTEMP
    cd $MIRTEMP
    tar -xf ~/miriad.tar
    set path = ($MIRTEMP/src/scripts $path)
    rehash
    mir.mkdirs $MIRTEMP
\endtt}
Set up the {\tt MIRRC} file.  This file will be ``sourced'' to set user
environment variables.  You should read the comments in the file before
proceding.  {\tt \$MIR} is the home directory of the MIRIAD tree; 
{\tt \$MIRHOST} is the platform on which you are installing MIRIAD (either
{\tt sun3} or {\tt sun4} or {\tt convex}.  {\tt \$MIRXLIB} and
{\tt \$MIRXINC} are intended to allow you to specify library and include
directories (respectively) that are to be searched, if these are not in the
``standard'' places.
{\ninepoint\begintt
    cp $MIRTEMP/src/scripts/MIRRC.skel $MIRTEMP/MIRRC
    cd $MIRTEMP

    [edit MIRRC: the changes for this example are]

          setenv MIR     /home/babbage/miriad/miriad.home
          setenv MIRHOST "sun4"
          setenv MIRXLIB ""
          setenv MIRXINC ""
\endtt}
Set up the controlling file for users.  You should read the comments in the
file before proceding.
{\ninepoint\begintt
    cp $MIRTEMP/src/scripts/thisfile.skel ~/cshrc.MIRIAD
    cd ~
    [edit cshrc.MIRIAD] ... for this example, the executable lines of
                            the file should be

    if (`arch` == 'sun4' && -e /home/babbage/miriad/miriad.home/MIRRC) then
       source /home/babbage/miriad/miriad.home/MIRRC
       endif
\endtt}
Set up astronomy-related files.  These files control the maximum sizes of
images (eg, 512x512 pixels) that can be handled by MIRIAD.
\item{$\bullet$} {\tt MAXDIM} is the maximum dimension allowed for an
image (if set to 512, then 512 by 512 by DIM3 images can be dealt with; if
set to 2048, then 2048 by 2048 by DIM3 images can be dealt with; etc.).
Static allocation of a plane of data (of size {\tt MAXDIM} ** 2) is done in
certain programs, so swap space considerations on your machines do apply.
\item{$\bullet$} {\tt MAXBUF} is an ``internal buffer size'', governing
the maximum number of datapoints that can be read at once.  It should be
set no lower than 4 * {\tt MAXDIM} ** 2.
\item{$\bullet$} {\tt MAXANT} is the maximum number of antennas that a
MIRIAD dataset can have.
\item{$\bullet$} {\tt MAXCHAN} is the maximum number of channels that a
MIRIAD dataset can have.
\item{$\bullet$} {\tt MAXNAX} is the maximum number of axes allowed for a
MIRIAD dataset.
{\ninepoint\begintt
    [edit $MIRTEMP/src/inc/maxdim.h] ... set MAXBUF, MAXDIM, MAXANT, and
                                            MAXCHAN

    [edit $MIRTEMP/src/inc/maxdimc.h] ... set MAXBUF, MAXDIM, MAXANT, and
                                              MAXCHAN to be the same as
                                              above

    [edit $MIRTEMP/src/inc/xyzio.h] ... set XYZIOMAXBUFLEN to be the same
                                            as MAXBUF above.

    [edit $MIRTEMP/src/inc/maxnax.h] ... set MAXNAX
\endtt}

\beginsub{Installing MIRIAD}

Set up the MIRIAD account owner's {\tt .cshrc} file:  anywhere AFTER the
path has been set, insert the lines
{\ninepoint\begintt
    if (`arch` == "sun4" && -e ~miriad/cshrc.MIRIAD)   then
       source ~miriad/cshrc.MIRIAD
    endif
\endtt}
Then ``source'' the {\tt .cshrc} file:
{\ninepoint\begintt
    source ~miriad/.cshrc
\endtt}
Put the scripts into the binary directory and start the installation.  The
command {\tt mir.loadall \&} loads the entire system from source code
(running the job in background), and leaving log files in directory
{\tt \$MIR/tmp}.  These log files are discussed later in this chapter.
{\ninepoint\begintt
    cp -p $MIR/src/scripts/mir\.* $MIRBIN
    rehash
    mir.loadall &
\endtt}

\beginsub{Setting Up The User Community}

Each user should place the following lines in the {\tt .cshrc} file
anywhere AFTER the directory searchpath has been set:
{\ninepoint\begintt
    if (`arch` == "sun4" && -e ~miriad/cshrc.MIRIAD)   then
       source ~miriad/cshrc.MIRIAD
    endif
\endtt}
Then ``source'' the {\tt .cshrc} file:
{\ninepoint\begintt
    source ~miriad/.cshrc
\endtt}
After doing this, they are set up automatically at logon.

\beginsection {Printing the Manuals}

MIRIAD manuals are provided in {\TeX } and {\LaTeX } format, and a device
independent output file is generated (a {\tt .dvi}) file.  These are
converted to {\tt PostScript} files in the usual manner ({\it e.g.}, via
{\tt dvitps myfile.dvi > myfile.ps}).  Covers (front and back) for all the
manuals are in directory {\bf \$MIR/manuals/covers} in {\tt PostScript}
form.

If you don't have write access to MIRIAD's source tree, you'll need to copy
the contents of these directories to a directory of your own before you can
run either {\TeX } or {\LaTeX } on them.  The table below lists directory
locations relative to directory {\tt \$MIR/manuals}.  The two manuals that
you most need are the User's Guide (for using MIRIAD) and the Programmer's
Guide (for writing MIRIAD programs and incorporating them into your local
version of MIRIAD).  The rest may be of parenthetical interest to
programming, but have limited utility (on line documentation is far handier).
{\ninepoint\begintt
    MANUAL ................ DIRECTORY ... TO GENERATE PAPER
    User's Guide .......... uguide ...... make
    Programmer's Guide .... pguide ...... tex pg
    Executable Modules .... eguide ...... tex em
    Subroutine Library .... sguide ...... tex sl
    LINPACK Subroutines ... linpack ..... latex linpack.latex
\endtt}
In addition to the above manuals, two ``borrowed'' software packages have
been incorporated as-is into MIRIAD, and both come with manuals of their
own.  Browse through directory {\tt \$MIR/borrow/pgplot} for the PGPLOT
manual, and through directory {\tt \$MIR/borrow/wip} for the WIP manual.
Follow the instructions provided for generating these manuals (in particular,
the WIP manual is a genuine User's Guide for WIP).

\beginsection {Routine Maintenance}

MIRIAD is still in a late development phase (quite beyond a beta-release,
but still undergoing routine source code updates).  To facilitate these
updates, and since all existing MIRIAD sites have personnel with more than
passing familiarity with MIRIAD's internal structure, we are using an
explicit ``do this'' via scripts in order to recompile and relink programs.

We are scheduled to go to a ``formal release version'' in 1992, and to
thereafter put out regularly scheduled upgrades (a measure of stability in
the releases), but until that time comes, the only way to effectively
maintain MIRIAD with newly received code is to follow the same procedures
as do those who work with the system on a regular basis (even though that
may not be the case for you at this time).

\beginsub {Loading from Source Code}

MIRIAD consists of four subroutine libraries (LINPACK, PGPLOT, HDF, and
MIRIAD's own internal subroutines), as well as the executable tasks and
a few scripts used for maintenance purposes.  The entire system can be
loaded from source code with the command

\centerline{\tt mir.loadall \&}

This command runs the following scripts in order, leaving a logfile of
each in directory {\tt \$MIR/tmp} with the process id and the \$MIRHOST
name appended to it:
{\ninepoint\begintt
SCRIPT ........ LOGFILE ............. WHAT IT DOES
mir.scripts ... scripts.2846.sun4 ... Copy scripts from src to bin directory
mir.linpack ... linpack.2846.sun4 ... Compile LINPACK subs into library
mir.pgplot .... pgplot.2846.sun4 .... Compile PGPLOT subs into library
mir.subs ...... subs.2846.sun4 ...... Compile MIRIAD subs into library
mir.hdf ....... hdf.2846.sun4 ....... Compile HDF subs into library
mir.prog ...... prog.2846.sun4 ...... Compile/link tasks into bin directory
mir.wip ....... wip.2846.sun4 ....... Compile/link WIP into bin directory
mir.mxv ....... mxv.2846.sun4 ....... Compile/link MXV into bin directory
mir.gendoc .... gendoc.2846.sun4 .... Generate online docs for tasks/subs
\endtt}
You can run these scripts at any time to recompile and relink subroutines
and programs (the output will be sent to the screen unless you reroute it;
only the {\tt mir.loadall} script automatically routes output to log files).

For a fuller explanation of what can be done with these scripts, see the
MIRIAD manual for Executables.

\beginsub {Likely Problem Areas}

The sole likely area where you may have problems is in getting {\tt mir.mxv}
to compile and link; the reason is that it requires a large number of X
libraries beyond what is usually installed on a normal system.  In this case,
there is nothing you can do about it:  the libraries are currently required
(though we are in the process of reducing our reliance on them).

Once you have run {\tt mir.loadall}, you can quickly and easily check whether
everything went ok.
{\ninepoint\begintt
COMMAND .............................. WHAT IT DOES
cd $MIR/tmp .......................... cd to look at the logfiles
ls -ltr .............................. show logfiles in the order created
grep -i error * | more ............... show any errors
grep -i fail  * | more ............... show any failed compilations
grep -i ld  prog* wip* mxv* | more ... show any programs that didn't link
\endtt}
You'll need to deal with any problems that have arisen.  Scroll through the
logfiles ({\tt emacs}, {\tt vi}, or however) where you may suspect problems.

Next, enter {\tt ls -lt \$MIRBIN} and pipe the output to {\tt more}.  Check
whether all the binaries are executable (an {\tt awk} script is the only
file in the directory that is not executable).  This is more of a
confirmation of the previous step than anything else.

\beginsub {MIRIAD Scripts}

The list below shows all MIRIAD scripts and their aliases.  They are more
fully discussed in the Executables manual (more conveniently, see the on
line documentation).  If you are getting regular updates of MIRIAD source
code, then the only scripts you should need to use on a regular basis are
{\tt mir.subs} and {\tt mir.prog}.
{\ninepoint\begintt
mir.autocal ..... A-priori calibration using selfcal
Autocal ......... A-priori calibration using selfcal
mir.bug.csh ..... Send mail about bugs to appropriate person
mirbug .......... Send mail about bugs to appropriate person
mirfeedback ..... Send feedback to the appropriate person
\endtt}
{\ninepoint\begintt
mir.find ........ Search for files/dirs in MIRIAD source code
mirfind ......... Search for files/dirs in MIRIAD source code
mir.gendoc ...... Generate on-line MIRIAD documentation
mir.hcconv ...... Load MIRIAD program HCCONV from source to $MIRBIN
mir.hdf ......... Compile HDF source code (SUN only)
\endtt}
{\ninepoint\begintt
mir.help ........ Obtain programmer-useful help about MIRIAD source files.
mirhelp ......... Obtain programmer-useful help about MIRIAD source files.
mir.imcalc ...... Load MIRIAD program IMCALC from source to $MIRBIN
mir.index ....... List MIRIAD tasks and 1-line descriptions
mirindex ........ List MIRIAD tasks and 1-line descriptions
\endtt}
{\ninepoint\begintt
mir.linpack ..... Compile linpack source  (SUN only)
mir.loadall ..... Load MIRIAD subsystems from source code
mir.mkdirs ...... Make the directories for a MIRIAD system
mir.mkdirs1 ..... Internal subscript for script mir.mkdirs
mir.mkpdoc ...... Generate a *.doc file under certain conditions
\endtt}
{\ninepoint\begintt
mir.mpath.awk ... awk script to insert $MIRBIN in searchpath
mir.msss ........ Load MIRIAD's version of the AIPS' SSS (SUN only)
mir.mxas ........ Load MIRIAD's version of the AIPS' XAS (SUN only)
mir.mxv ......... Load MIRIAD's X-Visualizer (SUN only)
mir.pgdoc ....... Generate a *.sdoc file for a PGPLOT subroutine
\endtt}
{\ninepoint\begintt
mir.pgplot ...... Load the PGPLOT library from source code
mir.prog ........ Load MIRIAD tasks from source code to $MIRBIN
mir.prog1 ....... Internal subscript for script mir.prog
mir.scripts ..... Copy scripts from source to $MIRBIN
mir.sndex ....... List MIRIAD subroutines and 1-line descriptions
\endtt}
{\ninepoint\begintt
mir.subs ........ Compile MIRIAD subroutines into $MIRLIB/libmir.a
mir.tar2m ....... Update system from a MIRIAD tarfile
mir.tools ....... Load MIRIAD's tools from source code into $MIRBIN
mir.uvdisplay ... Load UVDISPLAY from source code to $MIRBIN
mir.uvhat ....... Load UNIX UVHAT from source code to $MIRBIN
\endtt}
{\ninepoint\begintt
mir.wip ......... Load the WIP task from source code into $MIRBIN
\endtt}

\beginsub {A Trivial Partial Tutorial}

You should not need to use most of the scripts that accompany MIRIAD,
aside from the ones discussed in installing the system, though you may
certainly do so (they are generally invoked automatically, or not directly
related to supporting the system).  In particular, you should not need to
consider LINPACK, PGPLOT, or HDF again; the most common source code updates
are within MIRIAD's own subroutine library, and minor mods to executable
tasks.  Following is a simple update of source code.

Suppose you receive an ``update tarfile'' of source code, containing
a new version of subroutine {\tt fitsio.for}, an update of task {\tt fits.for},
and a new task (call it {\tt newtask.for}).  A {\tt tar -tf tarfile} shows
{\ninepoint\begintt
src/subs/fitsio.for
src/prog/convert/fits.for
src/prog/analysis/newtask.for
\endtt}
To update the source (backing off old source code to the {\tt \$MIR/oldsrc}
directory for safekeeping) and update the binaries, do the following:
{\ninepoint\begintt
cd $MIR
mir.tar2m <tarfilename>
mir.subs fitsio.for
mir.prog fits newtask
\endtt}
Updating is done from the {\tt \$MIR} directory.  {\tt mir.tar2m} installs
the source code.  {\tt mir.subs} installs the updated {\.o} file in the
appropriate library, and {\tt mir.prog} relinks the two named tasks (there
is no need to include the {\tt .for} extension, though you may do so).
On-line documentation is updated by the scripts.  You're done.

Since you ``know'' that no MIRIAD task other than {\tt fits} uses subroutine
{\tt fitsio.for}, there is no need to relink all the MIRIAD tasks, and here
is the fly in the ointment:  when a subroutine is updated, all executables
that rely on that subroutine (however indirectly) must also be relinked.
If you do not know which subroutines are used by any given task, then the
only safe course left to you is to relink all the tasks:  run 
{\tt mir.loadall prog \&} and wait until the job is done.

The reasoning for this seeming overkill is simple and sound:  a {\tt makefile}
system requires all dependencies to be listed to be effective (and we are
still tweaking the system, meaning that this method is presently inconvenient
and error prone); a {\tt makefile} system requires the {\tt .o} files to be
present or created (this is avoidable with some effort); we are occassionally
altering subroutines that use other subroutines (meaning that the
{\tt makefile}s are in a constant state of flux; and lastly, those of us
working on MIRIAD in its final stages of development are generally familiar
enough with the system that we can live with things that ``naive'' users
would not want to tolerate ({\it e.g.}, I just ``know'' that {\tt fitsio.for}
is used by only one task).

We intend to go to a ``version number'' updating system in 1992 (the
assumption being that we have achieved a good level of stability), at which
time source code updates will include a specific sequence of steps for you
to use to more easily bring your binaries up to date after installing source
code updates.

\beginsection {Local Variations in the Setup}

There are no hard rules in how you want to set up your local MIRIAD system
(though, for the sake of passing around source code updates, the fewer
variations in structure, the better).  If variations are done, try to keep
them within the scope of files under your own control ({\it i.e.}, insulate
the users).  Below is an alteration of the aforementioned {\tt cshrc.MIRIAD}
file to allow a ``motd'' specific to MIRIAD to be printed on the user's
console at logon time, but not otherwise.
{\ninepoint\begintt
if (-e /home/babbage/miriad/home.miriad/MIRRC) then
                source /home/babbage/miriad/home.miriad/MIRRC
endif
if ($?TERM && ! $?MIRHELLO) then
        setenv MIRHELLO x
        echo ""
        echo 'MIRIAD [29 Apr 92] environment variables set.'
        echo ""
endif
\endtt}
The date of the most recent update to the local MIRIAD is printed out at
logon.  Note that the {\tt TERM} environment variable is checked so as not
to interfere with standard {\tt UNIX} commands like {\tt rsh}, which do not
work correctly when this ``motd'' is passed back instead of the correct
information.

A possibly more useful variation would be to set an environment variable
to point all MIRIAD users to a directory containing local test data that
you've developed ({\tt e.g.}, after sourcing the {\tt MIRRC} file, include
the line {\tt setenv MYMIRDAT /home/somedata/datadir}).  Users need do 
nothing themselves to be set up, and you can simply tell that that there is
testdata available in directory {\tt \$MYMIRDAT}.

\beginsection {Common Occurance}

It happens from time to time that system changes will be made that force the
recompilation of source code (for example, perhaps your {\tt FORTRAN}
compiler is going from version 1.3 to 1.4).  This upgrade forces you to
reload MIRIAD from source code, since binaries are not compatible across
the upgrade.

One way to do it would be to wait for the formal upgrade, shut down MIRIAD,
and reload the system (following the procedure outlined at the beginning of
the chapter:  you don't have to port it again, but you do have to run the
command {\tt mir.loadall \&} and wait for results).

A better way (if you have the diskspace available), is to ``port'' MIRIAD
to another directory on your own system, set yourself up to use the newer
{\tt FORTRAN} compiler, and install MIRIAD from source code.  When done,
you simply change MIRIAD's {\tt cshrc.MIRIAD} file to point to the new
MIRIAD system.  You can then pitch the old system (after convincing
yourself that users have logged off and logged back on).  This better 
insulates the users from disruptions (they never see the change; MIRIAD
keeps on working as before).

It also gives you the chance to see if there are any problems with the
upgrade, with sufficient time to deal with them while the ``old'' MIRIAD
is still installed.  If you wait for the upgrade before beginning, then
you'll be dealing with any problems on a ``fire'' basis, with users 
waiting for your re-installation so that they can resume work.

Finally, system managers are generally keen to have someone act as guinea
pig for upgrades prior to foisting the upgrade onto the community at
large.  Historically, MIRIAD has a record of well-exercising upgrades in
compilers, and in OS upgrades as well.
