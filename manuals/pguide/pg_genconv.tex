%------------------------------------------------------------------------
% General Conventions Chapter
%------------------------------------------------------------------------

Programs are developed in the usual manner, using the MIRIAD
subroutine library (\$MIRLIB/libmir.a) to resolve external references.

FORTRAN source code has a ``for'' filename extension (e.g., myprog.for),
and is processed by MIRIAD's FORTRAN preprocessor ({\tt ratty}) to
generate a source code file for input to a compiler (one must specify
the output filename or the output is written to standard output).

{\tt ratty} preprocesses a few language extensions into standard FORTRAN
and flags a few bad programming practices.  {\tt ratty} is discussed in some
detail in the chapter on Utility Programs.

MIRIAD tasks should be written in Fortran-77.  Though you should
program in standard FORTRAN, two extensions may be utilized by
the programmer. Firstly, MIRIAD uses both upper and lower case
character strings (Fortran-77 strictly supports only upper case characters.
However almost all compilers support both upper and lower case, and it
would be a reasonably simple preprocessing job to convert all of
MIRIAD to a strictly upper case system if the need ever arises).
Generally, MIRIAD routine names are case-sensitive, with lower case
being preferred.

Secondly, MIRIAD tasks should use the {\tt maxdim.h} and
{\tt mirconst.h} include files where appropriate.  The former defines a
parameter, {\tt maxdim}, which gives the maximum image dimension that a
task should be prepared to accept. Currently this is set to 512, but
by using {\tt maxdim} to define needed storage, it should be reasonably
easy to rebuild all MIRIAD tasks to handle larger images.  The
include file also defines a parameter {\tt maxbuf}, which is a guide to
the maximum amount of internal data storage that a program should contain.
The latter defines a number of commonly used constants (e.g., {\tt pi}) in both
single and double precision.  Both {\tt maxdim.h} and {\tt mirconst.h}
reside in the MIRIAD include directory, \$MIR/src/inc.

Despite the encouragement to use these include files, programmers are
generally discouraged from using include files and common blocks. This
is far from a strict rule, but avoid them if you can.

A step towards aiding in keeping track of the source is that a program,
as the very first thing it does, should print out a version identification.
The following is typical:
{\ninepoint\begintt
       program Clean
          .
          .
          .
       character version*(*)
       parameter(version='(Clean: version 1.0 26-jan-90)')
          .
          .
          .
       call output(version)
\endtt}
This gives the task's name (Clean), the current version number (1.0),
and the date when the last substantive modification was made
(January 26, 1990).  Optionally, the responsible programmer might append
personal initials after the date.  The version number should also go
into the history comments generated by the task.
