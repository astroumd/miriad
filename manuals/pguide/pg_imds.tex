%------------------------------------------------------------------------
% Chapter 9 - IMAGE DATASETS
%------------------------------------------------------------------------
%
%  History:
%
%   rjs  28mar90  First version for System Guide
%   mjs  22mar91  Adapted into Programmer's Guide
%------------------------------------------------------------------------
\beginsection{Image Datasets}

These routines access image data sets.
{\ninepoint\begintt
      subroutine xyopen(tno,dataname,status,naxis,nsize)
      subroutine xyclose(tno)
      subroutine xyread(tno,index,array)
      subroutine xywrite(tno,index,array)
      subroutine xysetpl(tno,naxis,nsize)
\endtt}
Here {\tt xyopen} opens the image data set, and readies it for reading or
writing. {\tt dataname} is the name of the data set, {\tt status} is either
{\tt `old'} or {\tt `new'}, depending on whether an old data set is being opened
to be read, or a new data set is being created. {\tt naxis} gives the dimension
of the {\tt nsize} array. {\tt naxis} is always an input parameter.
{\tt nsize} gives the size, along each axis of the
image. When opening an old data set, {\tt nsize} is filled in by {\tt xyopen},
and passed back to the caller. For a new data set, {\tt nsize} must be set
to the
size of the desired image before the open routine is called. The argument
{\tt tno} is the handle passed back by the open routine, and is used
in all subsequent calls to identify the data set.

{\tt xyclose} closes the data set.

{\tt xyread} and {\tt xywrite} read or write a single row of the image. 
The row number is given by {\tt index}, and the pixel data is stored in
{\tt array} (a real array). By default, {\tt xyread} and {\tt xywrite} access
a row in the first image of a multi-image data set. The routine {\tt xysetpl}
is used to change this default to another image. In this, {\tt naxis} gives the
dimension of the {\tt nsize} array, and {\tt nsize} is an integer array
giving the indices along the third, fourth, etc, dimension of access. For
example, to access the n'th image in a cube, use:
{\ninepoint\begintt
      call xysetpl(tno,1,n)
\endtt}

\beginsection{Image Coordinate System}

MIRIAD defines and stores image coordinate system information in a similar
fashion to AIPS and FITS. Most cubes will have coordinates along its three
axis of RA, DEC and velocity. The item {\tt ctype} gives the
type of coordinate along a particular axis, whereas {\tt crval, crpix} and
{\tt cdelt} give the value of the coordinate at the reference pixel, the
value of the reference pixel, and the increment between pixels, respectively.
Unlike AIPS and FITS, RA and DEC are given in radians, and velocity is given
in km/sec. RA and DEC will generally need to be converted to
hours-minutes-seconds, or degrees-minutes-seconds, before being presented
to the user.  Subroutine {\tt deghms} can be used to perform this function.

The following table gives quite approximate formulae for converting from
pixel number to an astronomical coordinate. For more accurate formulae, see
AIPS Memo No. 27, ``Non-linear Coordinate Systems in AIPS'' (Eric Greisen).

\vskip .2in
{\raggedright
\def\bbox#1{$\underline{\smash{\hbox{#1}}}$}
\ninepoint
\tabskip=0em
\halign {#\tabskip=1em&#\tabskip=1em&#\tabskip=1em&#\tabskip=1em&#\hfil\cr

\bbox{\it Ctype}&
\bbox{\it Crval}&
\bbox{\it Crpix}&
\bbox{\it Cdelt}&
\bbox{Equation}\cr
\cr
{\tt RA---xxx}&
$\alpha_0$&
$i_0$&
$\Delta\alpha$&
$\alpha = \alpha_0 + \Delta\alpha (i-i_0) / \cos(\delta_0)$\cr

{\tt DEC--xxx}&
$\delta_0$&
$i_0$&
$\Delta\delta$&
$\delta = \delta_0 + \Delta\delta (i-i_0)$\cr

{\tt VELO-xxx}&
$v_0$&
$i_0$&
$\Delta v$&
$v = v_0 + \Delta v (i-i_0)$\cr

Others&
$x_0$&
$i_0$&
$\Delta x$&
$x = x_0 + \Delta x (i-i_0)$\cr
}
}

For example, the following code fragment calculates RA, DEC and velocity.
{\ninepoint\begintt
      character ctype1*8,ctype2*8,ctype3*8
      double precision crval1,crpix1,cdelt1,crval2,crpix2,cdelt2
      double precision crval3,crpix3,cdelt3
      double precision alpha,delta,v
      integer i,j,k
          .
          .
      (assume i,j,k contains the grid coordinate of interest)
          .
      call rdhda(tno,'ctype1',ctype1,' ')
      if(ctype1(1:5).ne.'RA---')
     *    call bug('f','First axis is not RA')
      call rdhda(tno,'ctype2',ctype2,' ')
      if(ctype2(1:5).ne.'DEC--')
     *    call bug('f','Second axis is not DEC')
      call rdhda(tno,'ctype3',ctype3,' ')
      if(ctype1(1:5).ne.'VELO-')
     *    call bug('f','Third axis is not velocity')
      call rdhdd('crval1',crval1,0.d0)
      call rdhdd('crval2',crval2,0.d0)
      call rdhdd('crval3',crval3,0.d0)
      call rdhdd('crpix1',crpix1,1.d0)
      call rdhdd('crpix2',crpix2,1.d0)
      call rdhdd('crpix3',crpix3,1.d0)
      call rdhdd('cdelt1',cdelt1,1.d0)
      call rdhdd('cdelt2',cdelt3,1.d0)
      call rdhdd('cdelt3',cdelt3,1.d0)
      alpha = crval1 + cdelt1/cos(crval2)*(i-crpix1)
      delta = crval2 + cdelt2            *(j-crpix2)
      v     = crval3 + cdelt3            *(k-crpix3)
\endtt}

This code fragment checks that the  axis are in the order RA, DEC then velocity
(i.e. the normal ordering)
and aborts if they are not. Smarter code would allow them in any order, and
would probably treat any unrecognized {\tt ctype} as a linear coordinate system.
It uses default values (if values of {\tt crval}, {\tt crpix} and {\tt cdelt}
are missing) of 0, 1 and 1. Probably better default values could be chosen,
although if {\tt ctype} is an item, then we can be fairly certain that the
other parameters will also be present.

\beginsection{Region of Interest and Pixel Blanking}

It is a common situation for the user to want to process a limited subset
of an image. It is also common for an image to contain pixels which are
``blanked'' or have an ``undefined value''. MIRIAD routines are available to
treat these two pixel selection operations together. The input to these
routines are the task parameters which describe the subregion the user
is interested in, and a masking item that may be associated with an image.
The output is a description of the pixels selected.

Each image dataset may have a masking item. This
is a bitmap containing a flag for each pixel, indicating
whether the pixel is good or bad. For bad pixels, the pixel value actually
stored
in the image is not defined, though it will be a legal or typical value.
Zero, or the value of the pixel before blanking, is a good choice.

\beginsub{Regular Regions of Interest}

Generally a minimum of three routines are needed to process even ``regular''
regions of interest. By ``regular'' we mean that the region of interest is
describable by a bottom left corner and top right corner (i.e., a rectangular
region). The routines of interest are:
{\ninepoint\begintt
        subroutine boxinput(key,dataset,boxes,maxboxes)
        subroutine boxset(boxes,naxis,nsize,flags)
        subroutine boxinfo(boxes,naxis,blc,trc)
\endtt}
All routines require an integer array, {\tt boxes}, which is used to
accumulate a description of the region of interest. Its size is given
by argument {\tt maxboxes}. The required size is a function of the
complexity of the region of interest, etc. Typically 1024 elements
should be adequate.

{\tt boxinput} reads the task parameters that the user gives to
specify the region of interest. The way this is specified is moderately
general and (consequently) complex. See the description in the users guide.
Included is the ability to give the region specification in a variety of
units. To convert between some units and absolute pixels, {\tt boxinput}
needs information about the coordinate system being used (e.g.
parameters {\tt crval}, {\tt crpix} and {\tt cdelt}). {\tt boxinput} determines
these from the MIRIAD data-set given by {\tt dataset}. Normally this is the name
of the image dataset which we are interested in. If {\tt dataset} is
blank, {\tt boxinput} still functions correctly, but cannot perform unit
conversion.

{\tt boxinput} breaks up the specification into an intermediate form, and
stores it in the {\tt boxes} array.
The keyword associated with the task parameter is {\tt key},
which would normally be {\tt 'region'}. If the {\tt key} argument is blank,
{\tt boxinput} does not attempt to get a region-of-interest specification,
but instead just initializes the {\tt boxes} array with the default
region-of-interest.

The programmer passes to {\tt boxset} information about the size of the image
of interest. This is given in the integer array {\tt nsize}, which consists
of {\tt naxis} elements (as with the corresponding arguments to {\tt xyopen}).
The {\tt flags} argument is an input character string, giving information
about the default region of interest. It can consist of:
\item{$\bullet$} {\tt q} The default region of interest is the inner
quarter of the image.
\item{$\bullet$} {\tt 1} The default region of interest consists of
the first plane only.
\item{$\bullet$} {\tt s} The region of interest must be ``regular''. If
it is not, {\tt boxset} generates a warning message, and will use the
bounding box of the selected region.

{\tt boxinfo} returns integer arrays {\tt blc} and {\tt trc}, which give the
bottom
left corner and top right corner of the region which encloses the overall
region of interest. Both these arrays are of size {\tt naxis} integers.

\beginsub{Arbitrary ROI and Blanking Information}

In addition to the above routines, there are three routines to allow treatment
of more complex regions of interest. These are:
{\ninepoint\begintt
        subroutine boxmask(tno,boxes,maxboxes)
        logical function boxrect(boxes)
        subroutine boxruns(naxis,nsize,flags,boxes,
    *        runs,maxruns,nruns,xblx,xtrc,yblc,ytrc)
\endtt}

{\tt boxmask} requests that an image mask be ANDed with the region
requested by the user. The input to the routine is the image dataset handle,
{\tt tno}, whereas the {\tt boxes} array is an input/output integer array
used to accumulate region of interest information. {\tt boxmask} can be
called multiple times, each time ANDing in a new image mask. This is an optional
routine. Typically it would be called for each of the input images, so that
``blanked'' pixels in the input images would be excluded from the region of
interest.

{\tt boxrect} returns {\tt .TRUE.} if the region
of interest is rectangular (i.e. whether the region of interest is entirely
described by {\tt blc} and {\tt trc}.

{\tt boxruns} returns the region selected for a particular plane.
The input arguments {\tt naxis} and {\tt nsize} are analogous to the same
arguments in the {\tt xysetpl} routine. {\tt runs} is
an integer array of size $3\times maxruns$. On output it indicates which
pixels in the plane are to be processed.  {\tt runs} consists of {\tt nruns}
entries of the form:
{\ninepoint\begintt
     j,imin,imax
\endtt}
This indicates that pixels (imin,j) to (imax,j) are to be processed.
All entries are non-overlapping.
There may be zero or many  entries for a particular value of j. The
table is in increasing order of j and imin. On output, the integers
{\tt xblc, xtrc, yblc}, and {\tt ytrc} give the bottom left and top right
corners, in x and y, of the smallest subimage which contains the selected
pixels.  The {\tt flags} argument, an input character string, indicates
some extra options. These are:
\item{$\bullet$} {\tt r} Make the coordinates, returned in the {\tt runs}
table, relative to {\tt (xblc,yblc)}.

\beginsub{Reading and Writing Blanking Information}

Though the {\tt box} routines are the preferred way to read blanking
information, it is possible to read the blanking information associated with
an image directly. Also a routine is needed to write blanking information.
Blanking information can be read and written by two sets of routines.
These routines are:
{\ninepoint\begintt
      subroutine xyflgrd(tno,index,mask)
      subroutine xyflgwr(tno,index,mask)
      subroutine xymkrd(tno,index,runs,n,nread)
      subroutine xymkwr(tno,index,runs,n)
\endtt}
Here, {\tt tno} is the image handle returned by {\tt xyopen}, {\tt index}
gives the row number (analogous to {\tt xyread} and {\tt xywrite}).
Analogous to the working of {\tt xyread} and {\tt xywrite}, the plane of the
masking file that the routines access is set with the {\tt xysetpl} routine.

The {\tt xyflgrd} and {\tt xyflgwr} routines read and write the logical
array {\tt mask}. The {\tt mask} array has a {\tt .TRUE.} value if the
corresponding pixel is good, or {\tt .FALSE.} if it is bad (or blanked).

The {\tt xymkrd} and {\tt xymkwr} routines read and write a ``{\tt runs}''
array. The {\tt runs} array is a table of the form:
{\ninepoint\begintt
    imin,imax
\endtt}
where pixels {\tt imin} to {\tt imax} are good (not blanked). Note that
the size of {\tt runs} is {\tt n} integers of {\tt n/2} pairs of
ordinates. {\tt runs} is input to {\tt xywrite} and output from
{\tt xyread}. For {\tt xyread}, {\tt nread} returns the number of
elements of {\tt runs} that have been filled in.

Because many images are completely good (i.e. no blanked pixels),
it would be superfluous to always carry around blanking information.
Hence the
mask containing the blanking information need not exist. If it does not
exist, the read routines described above will return indicating that all
pixels are good.
A programmer can check if the mask exists, using the logical function
{\tt hdprsnt}:
{\ninepoint\begintt
     logical exists
     logical hdprsnt
          .
          .
          .
     exists = hdprsnt(tno,'mask')
\endtt}
