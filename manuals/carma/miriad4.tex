%\documentclass{report}
\documentstyle[11pt,aaspp4]{article}
\slugcomment{CARMA memo xxx - 17jun2011}

%\documentclass{article}

\begin{document}
\title{
{\Large  {\bf CARMA MIRIAD 4.2} } \\
{\footnotesize a.k.a. MIRIAD V4, a.k.a. the ABC's of MIRIAD}}
\medskip

\author{ Teuben, P.J. and Wright, M.C.H.}
\affil{University of Maryland, College Park, MD 20742}
\affil{Radio Astronomy laboratory, University of California, Berkeley,
    CA, 94720}

%\maketitle

\begin{abstract}
 
We summarize the current status of MIRIAD, in particular efforts that
have been going on since the introduction of CVS for code
management in early 2001.  Miriad
is routinely used for BIMA, OVRO, ATNF and WSRT data reduction.  In
collaboration with WSRT and ATNF several modules for the analysis of
data from these arrays have been added to MIRIAD.  MIRIAD is being used
for ALMA, ATA, CARMA, and LOFAR simulations.  Support for more than 256
antennas and for datasets larger than 2GB (8GB upcoming) has been added
to enable ATA and LOFAR simulations. Other new directions include
the WSRT parallelization efforts and a python interface to miriad.
The situation with the ATNF version is discussed as well, and differences
between the two versions are highlighted.


\end{abstract}

\section{Introduction}

The initial MIRIAD development occured between 1987 and 1990 for BIMA,
in Illinois, with help from partners in Berkeley and Maryland.
After Sault departed Illinois in May 1990, development of 
miriad continued in two separate versions: the A (ATNF, initially
Sault) and B (BIMA,  first at Illinois, now at Maryland). An attempt was made
to merge the two versions,
as presented in a proposal by Teuben at a BIMA board meeting in 2000 (?), but
the ATNF group preferred to continue their own development. This proposal
in turn was triggered by the departure of the then miriad maintainer
Doug Roberts, which left no formal support for Miriad maintenance at NCSA. 
The flat tree and RCS-based checkin/checkout system was 
inflexible, and adding new files was very complicated.
Early in 2001 the situation became critical (software rot etc.), and 
Maryland decided to take the RCS
system and import it into CVS for a more modern approach to shared
development. which was dubbed Version 3 of Miriad.

This version had a 2 year development cycle, which nominally ended February 2003,
when the Version 4 branch was merged into the mainline development. 
In practice this meant that Version 3 would
only obtain bug fixes for as long as we need it. In January 2004 Version 3
was closed, since data could now be reliably transferred between
Miriad3 and Miriad4.

We now consider Miriad4 the "C" release (C for CARMA), which also signals the 
manual merging of as much ATNF specific code as can be managed, 
as well as Oosterloo's WSRT code (the initial
impetus to solve the 2GB problem came from Tom Oosterloo at the WSRT where Miriad
is now used routinely).


\section{MIRIAD1 and MIRIAD2}

Here is a brief chronology of the development of Miriad. Version numbers were not
really as well established as they are from 3.0 onwards.

\begin{enumerate}
\item[1987] Miriad ("V1") developed by Sault, largely based on his experiences
with WERONG. Initially mainly dealing with visibilities, and a simple image
interface. At that time GIPSY was thought to be the image analysis package,
though fairly soon image analysis routines were added to Miriad.

\item[1990] Bob Sault moved to ATNF, and together with Neil Killeen were
responsible for a substantial portion of development, and a large number
of new and improved tasks (some ATCA specific).

\item[1990] Mark Stupar overhauls the directory structure, much to Sault's
chagrin (the basis of the later split between the ``A'' ``B'' versions of
MIRIAD.  We call this Miriad V2.

\item[1995] flat tree RCS archive in Illinois (Doug Roberts)

\item[1997] mini (C) version of the miriad library, as used in AIPS++, was extracted. 
Later (see below) this was coded in ANSI-C, for portability. This "mini"
library was code-wise identical to the one in miriad.

\item[2001] Miriad V3 born out of CVS, coincides with the move of development from
NCSA to UMD.

\item[2003] Miriad V4 with files larger than 2GB and 8GB. 
Also dubbed the CARMA release.

\item[2004] Miriad V3.1.1 (Released as 04jan13) was the last official
Miriad V3.

\item[2011] Miriad V4.2 (to be released before 2011 Summerschool) with
WCSLIB merged in as a required package. 


\end{enumerate}

\section{MIRIAD3}

Here is a list of the accomplishments of MIRIAD V3. We have
released Version 3.1.1, which is now off a branch (-r MIR3) since MIRIAD V4 is
now at the head of development. Only bug fixes are applied to this version
of MIRIAD.

\begin{enumerate}

\item
Miriad source code imported into CVS

\item
Minor bug fixes, minor feature additions,
and a few new program here and there (e.g. {\tt deproject, rotcurmask})

\item
(some of it done before?)
antenna could be 1..256, since baselines were counted (256*A1+A2).  Already the
ATA needed about 500, so the formulae was changed

\item
fighting with the SONG team to keep the software in sync (not 100\% successfull yet)

\item
started to integrate ATNF specific code (therefor loosing the need for two
versions of miriad?). We may have to keep up with some of the new things
in the ATNF release, since they have expressed no real interest in
merging the versions.

\item
...

\end{enumerate}


\section{MIRIAD4}

In the following list only a few items have been 
implemented\footnote{listed with a *}, most is just an incomplete wish list.
\begin{enumerate}
\item
website needs to be simplified. currently on \newline
\verb+http://www.astro.umd.edu/~teuben/miriad/+
and only linked in via \newline
\verb+http://bima.astro.umd.edu/miriad/+ where it should be.

\item
$\ ^{*}$
pyramid: a simple python interface

\item
$\ ^{*}$
last remaining remnants of the old (baseline=256*A1+A2) formulue, and introduction
of the new formulae that can handle $>$ 2048 antennae. ????   Eric Greisen chimed
in. Can't use $>$ 256 due to random groups convention (well, maybe 2047?). But
our fits program can't handle $>$ 256 for now. Internally we can do 2048 (for now),
though there is a suggested formulae to change this to 32768, the absolute
maximum for an integer*4.

\item
are we sure LFS is working for fits files?? needs to be tested  (since AIPS++
just ran into that, i began to wonder too....)

\item
MAXBUF  could be bigger? better optimized? configure (see below) will do this
eventually.

\item
$\ ^{*}$
The split MAXDIM/MAXDIM2, MAXANT/MAXANT2 is still not perfect... too many parameters
always will result into situations wehre it didn't work.

Also note there is some code for old vectorizing compilers/machines that use
compiler directives like \verb+c#maxloop 32+. For MAXANT $>$ 32 obviously this
will not do much good.

\item 
export of small variables into type safe locations (as in NEMO's outkeys= interface).

\item
should introduce an integer VERSION in the visibility data, just to make sure
that data in the future remains in lock with the software

\item
integer*8 support; ratty flag to go back to the old situation if
the fortran compiler does not support integer*8
[this will support scratchfiles $>$ 8GB]

\item
write out proper long integers in itemize
question is how fortran can do this?

\item
a conversion routine from MIR4 to MIR3 format (int8 $\rightarrow$ int) if possible
this program/script should check if all the int8 variables are short
enough that they can be cut into an int4:

\begin{verbatim}
	nwcorr
	ncorr
	vislen	(points to visdata, which does not need to be changed)
	
	? flags		?? could this ever be > 2GB ???
	? wflags
\end{verbatim}

\item
programs that read ascii files, tables and such should *as a rule*
skip lines that start with the (universal unix) comment symbol (\#)

\item
mir-config that can spit out the cflags, libs etc...
for example , in the current release uvio  is not linked with LFS,
unlike the rest. so uvio will hang....

\item
$\ ^{*}$
use scropen8,scrclose8,scread8,scrwrite8 in addition to scropen stuff.
this will simplify the amount of code that needs integer*8



\item
design a fftw interface to miriad's fft routines - should give an impressive
speedup, certainly on intel hardware

\item
the UVFITS random groups format should be replaced with the modern
FITS IDI format (used by VLBA, AIPS++ ?, SMA, etc.). Also related
to the 256 maximum number of antennae problem.

\item
autoconf; see MIR/install for a start on this.

\item
a better mir.test to excersize and certify a release

\end{enumerate}

\begin{verbatim}
ATNF specific tasks:

atlod    Convert an RPFITS file into Miriad uv format.
atfix    Apply various miscellaneous corrections to ATCA visibility data  
elevcor  Correct ATCA data for elevation-dependent gain
plboot   Set the flux scale of a visibility dataset given a planet obs.    ***


WSRT specific tasks:

OVRO specific tasks:
\end{verbatim}

\section{Data I/O}

The Miriad Benchmark memo introduced a simple but workable benchmark to excersize
the miriad (UV) I/O system to aid in understanding the scaleability. The
{\tt uvgen} task in particular can be estimated to calculate the amount
of time it takes for a snapshot to be written to disk.

\section{CVS}

Here are some technical notes on CVS usage in MIRIAD, including branching, tagging
and versioning:

\subsection{Branching}

Branching:  only
done for serious development that has to happen in parallel to ongoing
development, particularly when it's an experiment that may fail.
(some of these can also happen in a persons private sandbox that never gets
committed)

\begin{verbatim}


     cvs co miriad                # checkout a fresh copy, (tag it???)
     cvs tag -b ....              # branch it
     cvs update -r BTNAME
     
     ...
     cvs co miriad                # checkout another fresh copy
     cvs up -j BTNAME             # merge it, resolved conflict, build it, test it, etc.etc.
     cvs ci                       # check back all the changes
\end{verbatim}

\subsection{Tagging}

Tagging: normally done when a version changes (e.g. from 4.0.2 to 4.0.3)

Do this from a clean checkout (also to prevent that WIP and RAD gets tagged)

\begin{verbatim}

     cvs co miriad
     cd miriad
     edit VERSIONS         <- add the date that 4.0.2 is done, and mark 4.0.3 as "W.I.P"
     edit cvs.tags         <- add 4_0_2, which is what you're about to do
     cvs tag mir_4_0_2
     edit VERSION          <- change to the new one, 4.0.3 in this case
     cvs ci VERSION


\end{verbatim}

\section{Miriad 2003 telecon}

\subsection{Agenda}

\begin{verbatim}

I see our goals as:
  1.  Determine what enhancements to the Miriad might warrant an increased
        investment in development.
  2.  Determine if there is sufficient need for a face-to-face meeting to 
        coordinate such development.
  3.  Develop a prototype agenda for such a face-to-face meeting.

AGENDA

Below is a list of possible areas of development.  I recommend that for 
each topic we attempt to briefly address the following questions:

  o  What is the motivation for pursuing this topic?  How important is it? 
  o  Would the topic require fundemental changes to the Miriad software 
       infrastructure?  
  o  Are their obvious paths to achieving desired capabilities?
  o  Does the topic require more discussion?

There are quite a few topics here, so we can't get into too much detail.  
(A need for detailed discussion is perhaps motivation for a face-to-face.)  

  Topic:
  *  64-bit platform support
  *  Large file support (Miriad, FITS)
  *  Improved scripting with Python (enhancements to pyramid)
       - better integration with Miriad tasking
       - flexible interrogation of uv and image datasets
       - returning task output to scripting environment
       - integration of non-task software
  *  Parallel Processing
       - need framework?, 
       - transparent handling of platform dependence (e.g. parallel clean)
  *  Parallel I/O
  *  Improved Visualization
       - user's choice?  
       - interactions with tasks, scripting environment?
  *  Algorithm development
       - support for heterogenous primary beams
       - multi-scale deconvolution
       - new calibration techniques: use of WVR, joint solving for
         polynomials in gain curves.
  *  General evolution of Miriad processing environment?
       If we imagine Python allowing for the integration of other
       software or other modes of processing, how does that change the
       way we look at Miriad as a processing environment?  How do we
       plan for the future (e.g. multi-processing, GUIs, distributed
       processing)?




\end{verbatim}

\subsection{Minutes}

\begin{verbatim}

Miriad Development Working Group Minutes
2003-06-11 Wednesday 1pm ET 

Attendance:
  MD:  Peter Teuben, Stuart Vogel, Marc Pound
  IL:  Dick Crutcher, Ray Plante, Dave Mehringer
  UCB: Mel Wright

GOALS

  1.  Determine what enhancements to the Miriad might warrant an increased
        investment in development.
  2.  Determine if there is sufficient need for a face-to-face meeting to 
        coordinate such development.
  3.  Develop a prototype agenda for such a face-to-face meeting.

AGENDA

Below is a list of possible areas of development.  I recommend that for 
each topic we attempt to briefly address the following questions:

  o  What is the motivation for pursuing this topic?  How important is it? 
  o  Would the topic require fundemental changes to the Miriad software 
       infrastructure?  
  o  Are their obvious paths to achieving desired capabilities?
  o  Does the topic require more discussion?

There are quite a few topics here, so we can't get into too much detail.  
(A need for detailed discussion is perhaps motivation for a face-to-face.)  

  Topic:
  *  64-bit platform support
  *  Large file support (Miriad, FITS)
  *  Improved scripting with Python (enhancements to pyramid)
       - better integration with Miriad tasking
       - flexible interrogation of uv and image datasets
       - returning task output to scripting environment
       - integration of non-task software
  *  Parallel Processing
       - need framework?, 
       - transparent handling of platform dependence (e.g. parallel clean)
  *  Parallel I/O
  *  Improved Visualization
       - user's choice?  
       - interactions with tasks, scripting environment?
  *  Algorithm development
       - support for heterogenous primary beams
       - multi-scale deconvolution
       - new calibration techniques: use of WVR, joint solving for
         polynomials in gain curves.
  *  General evolution of Miriad processing environment?
       If we imagine Python allowing for the integration of other
       software or other modes of processing, how does that change the
       way we look at Miriad as a processing environment?  How do we
       plan for the future (e.g. multi-processing, GUIs, distributed
       processing)?

MINUTES

Dick explained the motivation for this meeting (failure of AIPS++ to
build a user community, examination of how to support high-end
processing, ...).  There were many comments about the need to be
science driven.  

Mel: On-the-fly, parallel processing will become increasingly
important for future telescopes with high data rates (e.g. ATA).  

Ray: We should assume that enhancements should not
diminish current capabilities nor require a major change in how users
use the package.  

We assumed all gathered had copy of agenda and the draft memo, "CARMA
MIRIAD".   Peter indicated that the agenda was a more general list
compared to the memo's wish list; the wish list items, thus, would be
covered in the agenda.

* 64-bit Platform Support

There is a build on IL's Itanium box, thanks to Peter, using 32-bit
compatability mode.  True 64-bit support should be straight-forward
and can be handled via compile flags.  No further discussion needed.

* Large File Support

There are two limits: 2 GB & 8 GB.  The latter requires integer*8
support in general and a minor change at the format level (dataset
size pointer; see compatibility issues below).  Solution is fairly well
understood.  Peter: some minor improvements supporting the EOD pointer
within the uvio layer would be useful.  

Compatibility Issues:  
  * Miriad4 will be able to read either Miriad4 or Miriad3 datasets.
  * Miriad4 will convert automatically to Miriad4 datasets in any task
      that creates new output datasets.
  * Ray: Are there any compatibility issues for tasks that do in situ
      updates (e.g. uvflag)? 
  * Miriad3 will be updated to at least detect Miriad4 datasets.  May
      attempt to deal with them if they are small enough; otherwise,
      fail.  

No further discussion needed.

* Improved scripting with Python

Desired capabilities:
       - better integration with Miriad tasking:
            automatic generation of task interfaces as Python functions
            Ray has investigated how to expose necessary metadata to Python 
       - flexible interrogation of uv and image datasets:
            to get info like how many channels in a window
            allows generic recipes to adapt to datasets
       - returning task output to scripting environment
            ex: getting stats from histo into Python variables.
            probably requires mods in task Fortran code
            could develop simple API that could be integrated into
              tasks piecemeal, as needed.  (not all tasks would need this.)
       - integration of non-Miriad task software
            e.g. using other Python modules like XML
                 accessing software from other astronomical packages. 

These are important for the Pipeline but are useful to desktop users
as well.  None of these are particularly difficult, but it would
require additional coordination and, most importantly, input from
users.  

Mel: Concern about how tightly we bind to Python.  What if something
     better comes along later?  

Peter: SWIG will help with this: define binding SWIG's interface
     meta-language, SWIG then can generate bindings for various
     supported scripting languages.  

* Parallel Processing

** Parallel I/O:

Two approachs:
1. Mel explained that parallel I/O can often be effectively achieved by
   strategic splitting of datasets, primarily by channel.  This solution
   requires no change to the format.  Possible downside is whether this
   rough cut approach limits the amount of effective parallism that can
   be achieved (compared to something more fine-grained).  Do we forsee
   splitting data in ways other than channel?
   
2. An alternative would be to convert Miriad datasets (Miriad5?) to HDF5
   format with transparent support for parallel I/O.  This format is
   well-supported and widely used, and substantial expertise resides at
   NCSA.  The downside is that it could have substantial costs in
   rewriting the Miriad I/O layer and to users in supporting backward
   compatibility.  

We recommend a two-pronged prototyping approach:  Test out how far we
can go with the rough-cut approach, and to do a study in collaboration
with the HDF group to understand how Miriad data might be stored in
HDF5.  

[Ray notes:  we expect to start prototyping the former approach in the
Pipeline.  It makes sense to do the latter at NCSA as well.]

** Parallel Processing

What strategies do we want to apply?  DIY embarrasingly parallel (EP) at
the script level?  fine-grained parallelization with MPI?

What examples are there of parallelization that isn't fundementally
channel-based?  Mosaicing (of continuum sources), Gridding?

Ray: We have experience with MPI (from AIPS++), but we would like to
push the EP approach as far as we can before considering MPI.  MPI is
more intrusive to the fortran layer and is ultimately less flexible to
running in different parallel environments (e.g. clusters, Condor
pools, and service-based distributed computing).

* Improved Visualization

Ray:  Is visualization adequate in Miriad?  Have users been
sufficiently successful with current available choices (Karma, DS9,
Aipsview, Miriad visualization tasks)?  Where is interaction with
tasks and scripting environment important (e.g. creating clean boxes)?
Is it sufficiently well supported?

Peter: Less publicized capability for creating clean boxes using DS9.  

Peter: Current assessment of support for Miriad 
  DS9 -- image data piped from Miriad (using DS9's XPA mechanism)
  Karma -- reads Miriad natively (or via FITS?)
  Aipsview -- currently supports FITS and Miriad (via AIPS++ lib)
  IDL -- via FITS

Marc: we should not invest in re-implementing what already exists with
current apps.  

Peter: one area that is a bit crude still, IMHO, is WIP.  There
already exists a Python binding to PGPlot, so perhaps there is an
opportunity to improve things.  

Conclusion:  no fundemental changes necessary.  Incremental
improvements to enable new functionality using one or more of the
existing viz options may be helpful.

** IDL Support

There was a discussion thread on the prospects of supporting IDL.
Since there are a number of people within our community that use IDL,
there may be some benefit.  

Peter:  have resource at UMD in Stephen White.

Ray expressed concern about getting into a situation of supporting
duplicate functionality in IDL and Python.  

Peter did not think that there would be significant demand for
manipulating uv-data in IDL; however, there would certainly be
interest in reading Miriad images.  Given IDLs existing image display
and manipulation capabilities, supporting IDL need not be more than
providing a Miriad image reader.

* Algorithm Development

Dick: Question: How capable is Miriad as a development environment?  

Mel:  Experience has shown that it is quite capable as new tasks have
      been submitted by a variety of people.  The typical pattern for
      development is to take an existing task and modify it or use it
      as a template.  

Peter: The processing model is not as general as the Measurement
      Equation formalism used by AIPS++ (as the ME didn't exist when
      Miriad was developed).  There are some aspects of the library
      that might be considered expert knowlegde, but the capabilities
      are all there. 

Ray:  Two philosophical points (i.e. they may be important enough at
      this point to drive any changes to Miriad infrastructure):

      1)  The development pattern of modifying existing tasks may
          limit one's ability to try something fundementally
          different. 
      2)  One of features of AIPS++ that was meant to make new
          development easier is the exposing of low-level data and
          algorithm access to the scripting layer; this allows people
          to try out ideas in a rapid development environment before
          investing in a C++ implementation.  We are not envisioning
          supporting this in Miriad.  

Specific Capabilities:
** Support for heterogeneous primary beams

Ray: what currently exists? 
Mel: The basic functionality exists pb* routines support this.
     Primary beams are usually pre-defined.  There is one that is the
     geometric mean of an OVRO and BIMA beam; nothing currently does this
     combination automatically.  Could be improved.
Dick:  What about the case of different antennas looking at different
     fields?
Mel: Initially we thought about allowing for this; however, after some
     thought one realizes that there are a host of complicated, systematic
     errors one has to deal with.  The effort for handling these
     correctly may not be worth any increase in observing speed.  

** Multi-scale deconvolution

Mosmem attempts this in its own way; it's slow.

GILDAS has this functionality; can we use it (e.g. call it from
Python?)

The AIPS++ version is based on the GILDAS idea; however, it is
implemented differently.  Its implementation uses the standard AIPS++
decon tools to do the deconvolution on different scales.  The
coordinatation and combination of subsequent models are done in the
scripting layer (glish).  

Conclusion: there is user interest, but there are several approaches
that could be tried.  

** New Calibration techniques

use of WVR...
Mel:  when this was prototyped before, WVR data were converted to
corrections stored in a gains item.  Straight-forward from there.  

using optical pointing offsets to correct pointing errors...
Mel: best done on-line

decorrelation corrections...
[more words?]

Conclusion: while there do not seem to be any critical path
capabilities needed for CARMA, there is room for improvements and
experimentation with new techniques.  

* General Evolution of Miriad Processing environment.

** GUIs

Peter: experience shows that users want scripts more than GUIs.
       Python would allow users to experiment with simple GUIs.  Some
       people have created personal, custom GUIs as convenient way of
       running scripts with different inputs.  A little bit of glue
       for describing Miriad task (and script) inputs would make it
       easier to create GUIs.

Mel:   GUIs consider imp. for ALMA, according to the requirements.
       GUIs are helpful for guiding new users through a process; this 
       is considered an advantage of GILDAS.  

Ray:   There are different types of GUIs:
         1) High-level, user-oriented GUIs dedicated to a specific
            sequence of processing steps; often refered to as
            "wizards".  This level was never achieved with AIPS++, 
            though highly desired by users, according to their
            feedback. 

         2) General workflow GUIs.  This is an integrated GUI
            environement for stitching together and launching tasks.
            AIPS++ had this, but it was not well-liked.

         3) Custom GUIs:  highly specialized interfaces to a
            particular task.  xcorf is an example.  These are often
            costly to build.

         4) Quick-custom GUIs: these would be the simple GUIs whipped
            up by users, as Peter described.

       Given our experience with Users, there is not a lot of demand
       for 2).  Some experimentation could be done with 1) as needed.
       GUIs of type 3) will be written whenever needed, but not as a
       general practice.  

       I've started looking into the kind of glue Peter referred to
       as part of improving the integration of tasks in Python; it
       would also make building GUIs easier.  

Conclusion: the main priority right now is to provide just enough
support to allow experimentation (by both users and developers) in
Python.  Any coordinated effort toward GUIs should be user-driven. 

CONCLUSIONS/ACTION ITEMS

We have identified some areas that could use further developement.
For most of these, we have a sense of what level of effort is
warrented.  We have decided to collect these ideas and incorporate
them into Peter's draft note, transforming it into a white paper.
This will be distributed to our user community to get feedback.  

We will proceed to work on the white paper over email and possibly
another telecon.  It is likely that we would postpone a face-to-face
meeting until after the white paper is prepared and some feedback
gotten from users.  

Is Peter willing to be white paper editor?  Ray expects to be an
active contributer.

Action Items:
  Ray:    type up minutes and distribute.                  Done.
          collect corrections/additions
  Peter:  file minutes into rough draft in some form
            to start transformation to white paper.

\end{verbatim}

\subsection{Minutes: 17-feb-2004}

\begin{verbatim}

Some notes on the WSRT/UMD telecon on Feb 17, 2004:

Present: UMD: Peter Teuben, WSRT: Tom Oosterloo, Garrelt Mellema, John Romein

Since WSRT also still has a MIRIAD/ATNF version, and V4 is now the active CVS
version, they are looking into the last differences and making sure all the important
things have found their way into the new V4.

There appears to be a shadowing issue with WSRT data: (to quote Garrelt)

The final test for shadowing (in uvio) is identical in ATNF and V4, and so
was not changed since the branching off.
In the ATNF version this test flags the wrong antenna for WSRT, and a local fix was
introduced at WSRT (we call the ATNF version with all the local fixes the wsrtMiriad
version).
We still have to make 100% sure that the tests have identical (wrong) results in ATNF
and V4 (I would suspect so), in which case a fix for WSRT has to be introduced.

From my diff between wsrtMiriad (>) and V4 (<):
16) TAO changed this in WSRT version, but the bug(?) is still present in V4..
3410c3451,3459
<         if(u*u + v*v <= limit && w >= 0) return(1);

---
 > /*
 >    changed w>=0 into w<=0.
 >    with w<=0 the wrong antenna of the shadowed pair got flagged. E.g. if RTA
 >    was shadowing RT9, RTA got trown out!!  (TAO  25-Jul-02)
 > */
 >         if(u*u + v*v <= limit && w <= 0) {
 >            return(1);
 >         }
 >


Peter remarked that ``some tinkering wrt shadowing'' was done between Tam Helfer
and Peter, and he hoped this was not the cause of the WSRT findings. Peter and
Tam worked on MIRPROG/csflag.for for carma simulations, since they found it was
too hard to do it in uvio.c No changes to uvio were made for the Helfer/Teuben 
work on CARMA simulations.

WSRT wanted to know if the MAXDIM tinkering was done. Peter commented that yes,
most, if not all, problems were solved (uvplt the last remaining one), but
WSRT remarked that they had a dataset that just didn't fit in the default buffers
for the task uvspec (selecting only one stokes parameter made the task run).
So perhaps some more careful checking with ``standard datasets'' (see below) is 
needed, for both WSRT and BIMA.

WSRT is equally interested in advanced python use, along the lines of the memo
that Ray Plante sent around.  WSRT commented that Omegacam is also using python
for parallization, in particular built around ESO's eclipse image processing
library.

Peter commented that a simple parall. could be achieved (if only by demonstration)
with a shell script splitting the  dataset in N pieces, sending it via ssh
to machines that have NFS access and putting it back together. Tom commented
that they found the TSYS information in a dataset doesn't get split correctly
and perhaps more, thus making this seemingly simple demo more tricky than
thought at first glance.

WSRT was curious to the FFTW efforts. Peter said nothing much more than his
initial efforts were done. Like ATLAS, it has to be shown via profiling and
benchmarks where the bottlenecks are. Peter claimed that ATLAS will not be
interesting for miriad, but FFTW will.

WSRT commented that JIVE has also shown an interest in miriad, but that we (and they)
are all aware of basic and important missing functionality (fringe fitting). 

The dutch LOFAR group has indicated they may want to use miriad for
developing prototypes for parallelization of the selfcal process for
LOFAR.

Action items:

- profiling (WSRT will work on this)
- standard datasets for BIMA and WSRT observing modes, and an easy way to
  grab them via a curl or wget into e.g. $MIRDATA (see $MIR/data, which was
  recently created)
- manuals. We (BIMA and WSRT at least) should work on a chapter in the
  ``australian manual'' (MIR/manuals/atnf-user). Note they use a scheme
  where each chapter has its own page numbering, so not to disturb the
  rest of the manual.


\end{verbatim}

\section{Miriad 2011 telecon}


\section{Miriad 2011 telecon}
\begin{verbatim}

Location: Berkeley
Present:   Wright, Teuben, McWilliams, 

\end{verbatim}


\section{MIRIAD V4.2}

The major changed for version 4.2 are:

\begin{itemize}
\item
WCSLIB
\item
better large file support, e.g. FITS files over 8GB.
\item
memory management and ptrdif now merged from ATNF version
\item
example MIS pipeline (separate CVS module)

\end{itemize}

\subsection{CARMA-ATNF differences}

Despite various attempts to combine the two versions of MIRIAD, there 
remain two different versions. For the 4.2 release a renewed effort
took place to minimize these differences, including support to build
a separate ATNF version with the CARMA environment of MIRIAD.

More to come here.

\section*{References}

[1] WERONG - Sault's thesis? 1986?
\newline
[2] Sault R.J., Teuben P.J., Wright M.C.H., 1995,
 ``A retrospective view of Miriad'' in 
Astronomical Data Analysis Software and Systems IV, 
ed. R. Shaw, H.E. Payne, J.J.E. Hayes, ASP Conf. Ser., 77, 433-436.
\newline
[3] Teuben - BIMA memo 81 on miriad benchmarks (a ``live'' document)
\newline

\end{document}
