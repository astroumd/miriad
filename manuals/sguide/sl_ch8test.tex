\module{annotate}%
\newline \abox{File:} \$MIR/src/prog/analysis/implot.for
\newline \abox{Keywords:} plotting
\newline{\tenpoint
{\eightpoint\begintt
        subroutine annotate(tno,in,map,tmax,tmin,conlevs,numlevs,
     *                  sclo,schi,sdlo,sdhi)
        implicit none
        integer tno
        character*(*) in
        integer map
        real tmax, tmin
        real conlevs(50)
        integer numlevs
        real sclo,schi,sdlo,sdhi

  Annotate the map information to the right side of
  the page for ImPlot.

  Inputs:
    tno        The input image.
    in         The pgplot device.
    map        The map number in the image.
    tmax,tmin  Maximum and minimum of region of interest.
    conlevs    The array of contour levels.
    numlevs    Number of conlevs.
    sclo,schi,sdlo,sdhi  PGPLOT view port limits.
\endtt}
\par}
\module{beampat}%
\newline \abox{File:} \$MIR/src/prog/analysis/implot.for
\newline \abox{Keywords:} plotting
\newline{\tenpoint
{\eightpoint\begintt
        subroutine beampat(tno,corner,blc,trc,funits)

        implicit none
        integer tno,corner
        real blc(3), trc(3)
        character*(*) funits

  Add a picture of the beam to the specified corner of the plot.

  Inputs:
    tno        The handle ofthe input image.
    corner     is 0 for none or 1-4 to specify corner
    blc,trc    The corners of the region of interest.
    funits     Units of plot.
\endtt}
\par}
\module{beamsize}%
\noindent Find beam size and location.
\newline \ 
\newline \abox{File:} \$MIR/src/prog/deconv/tvcln.for
\newline \abox{Keywords:} image analysis
\newline{\tenpoint
{\eightpoint\begintt
        subroutine BeamSize(lBeam,n1,n2,xbeam,ybeam,bmaj,bmin,bpa)

        implicit none
        integer lBeam,n1,n2,xbeam,ybeam
        real bmaj,bmin,bpa

  Find beam size and location.
  Put bmaj, bmin, bpa into the map header if not present.

  Inputs:
    lBeam      Handle of the beam.
    n1,n2      Dimensions of the beam
  Outputs:
    xBeam,yBeam Location of beam peak.
    bmaj,bmin  fwhm beam size in radians
    bpa        position angle of beam in degrees.
\endtt}
\par}
\module{getconts}%
\newline \abox{File:} \$MIR/src/prog/analysis/implot.for
\newline \abox{Keywords:} plotting
\newline{\tenpoint
{\eightpoint\begintt
        subroutine Getconts
        implicit none

  Read the input items conflag and conargs and fills in 
               the common block contour -conflag and conargs(nconarg)
               (this must be put between keyini and keyend)
\endtt}
\par}
\module{getvaxis}%
\newline \abox{File:} \$MIR/src/prog/analysis/imspect.for
\newline \abox{Keywords:} image analysis
\newline{\tenpoint
{\eightpoint\begintt
        subroutine GetVaxis(lIn,vaxis)

        implicit none
        integer lIn,vaxis

  Find the v-axis.

  Inputs:
    lIn        The handle of the image.
  Output:
    vaxis      The velocity or frequency axis.
\endtt}
\par}
\module{imheader}%
\noindent Get image items: object, restfreq and date-obs.
\newline \ 
\newline \abox{File:} \$MIR/src/prog/analysis/imspect.for
\newline \abox{Keywords:} image analysis
\newline{\tenpoint
{\eightpoint\begintt
                Subroutine ImHeader(lIn,object,restfreq,date)

        implicit none
        integer lIn
        character*9 object,date
        double precision restfreq

  Get image header items 'object' 'restfreq' and 'date-obs' from image.

  Input:
    lIn        The handle of the Image.
  Output:
    object     The object or source name.
    restfreq   The restfreq of the observations, in GHz.
    date       The date of the observations.
\endtt}
\par}
\module{labels}%
\newline \abox{File:} \$MIR/src/prog/analysis/implot.for
\newline \abox{Keywords:} plotting
\newline{\tenpoint
{\eightpoint\begintt
        subroutine labels(tno,blc,trc,flag,ablc,atrc,trout,label)

        implicit none
        integer tno,blc(3),trc(3)
        character*(*) flag,label(2)
        real ablc(3),atrc(3),trout(6)

  Make the coordinate labels for plotting a Miriad image.

  The types of units are permitted are:
       p  pixels with respect to center
       s  arcseconds with respect to center, km/s, or Ghz
       a  absolute coordinates RA(hrs), DEC(degs), km/s, or Ghz

  Inputs:
    tno        The handle of the input Image.
    blc        The input bottom left corner.
    trc        The input top right corner.
    flag       The flag for coordinate labels.
  Outputs:
    ablc       User units bottom left corner
    atrc       User units top right corner
    trout      The translation array for pgplot
    label      The axis labels for the plot
\endtt}
\par}
\module{listdata}%
\noindent List Image in specifed format.
\newline \ 
\newline \abox{File:} \$MIR/src/prog/analysis/imlist.for
\newline \abox{Keywords:} image analysis
\newline{\tenpoint
{\eightpoint\begintt
        Subroutine ListData(lIn,naxis,blc,trc,fldsize,format)

        implicit none
        integer lIn,naxis,blc(naxis),trc(naxis),fldsize
        character format*(*)

   List Image in specified format into LogFile.

  Inputs:
    lIn        The handle of the Image.
    naxis      Number of axes of the Image.
    blc,trc    Corners of region of interest.
    format     Format specification.
    fldsize    Field size.
\endtt}
\par}
\module{listhead}%
\noindent List Image Header variables in a standard format.
\newline \ 
\newline \abox{File:} \$MIR/src/prog/analysis/imlist.for
\newline \abox{Keywords:} image analysis
\newline{\tenpoint
{\eightpoint\begintt
                Subroutine ListHead(tno)

        implicit none
        integer tno

  Read Image header variables.
  convert units and write in standard format into LogFile.

  Input:
    tno        The handle of the Image.
\endtt}
\par}
\module{liststat}%
\noindent List Image Statistics in a standard format.
\newline \ 
\newline \abox{File:} \$MIR/src/prog/analysis/imlist.for
\newline \abox{Keywords:} image, statistics
\newline{\tenpoint
{\eightpoint\begintt
        Subroutine ListStat(lIn,naxis,blc,trc)

        implicit none
        integer lIn,naxis,blc(naxis),trc(naxis)

   List Image Statistics and write in standard format into LogFile.

  Inputs:
    lIn        The handle of the Image.
    naxis      Number of axes of the Image.
    blc,trc    Corners of region of interest.
\endtt}
\par}
\module{maplabel}%
\newline \abox{File:} \$MIR/src/prog/analysis/implot.for
\newline \abox{Keywords:} plotting
\newline{\tenpoint
{\eightpoint\begintt
        subroutine MapLabel(tno,map)

        implicit none
        integer tno,map

  Label the map with the velocity for Implot

  Inputs:
    tno        The handle of the image.
    map        The map number.
\endtt}
\par}
\module{setconts}%
\newline \abox{File:} \$MIR/src/prog/analysis/implot.for
\newline \abox{Keywords:} plotting
\newline{\tenpoint
{\eightpoint\begintt
        subroutine setconts(isneg,fmax,fmin,conlevel,numcon)

        implicit none
        logical isneg
        real fmax,fmin
        integer numcon
        real conlevel(50)

  Setconts uses the common block contour -conflag and 
               conargs(nconarg) and its arguments to create a list of
               contours.
               conflag is made up of one letter codes which mean:
               p means the contour values are percentages of the maximum.
               i means the contour values are a list of contour values.
               a means the contour values are absolute numbers.
               if i is not present, the first contour value is used as a
               step to compute the other values between the min and max of
               the map.
  Inputs:
    isneg      true means  get negative contours
    fmax,fmin  Maximum and minimum values in image.
  Outputs:
    conlevel   Array of contour values.
    numcon     The number of conlevel.
\endtt}
\par}
\module{uvheader}%
\noindent Get uv variables: source, restfreq and date.
\newline \ 
\newline \abox{File:} \$MIR/src/prog/vis/uvplt.for
\newline \abox{Keywords:} uvdata analysis
\newline{\tenpoint
{\eightpoint\begintt
                Subroutine UvHeader(lIn,source,restfreq,date)

        implicit none
        integer lIn
        character source*9,date*18
        double precision restfreq

  Get uv variables source, restfreq and time from uvdata.
  Convert time to calendar date and time.
  This routine must be called after a call to uvread.

  Input:
    lIn        The handle of the uvdata.
  Output:
    source     The object or source name.
    restfreq   The restfreq of the observations, in GHz.
    date       The date and time of the observations.
\endtt}
\par}
